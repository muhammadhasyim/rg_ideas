%===============================================================================
\chapter{Strongly Correlated Electrons: The Hubbard Model}
\label{ch:hubbard}
%===============================================================================

The Hubbard model captures the essential physics of strongly correlated electron systems, where the interplay between electron kinetic energy and on-site repulsion produces a rich phase diagram. This chapter applies the six-step recipe of Chapter~\ref{ch:recipe} to a condensed matter system where strong correlations challenge perturbative methods and reveal the full power of RG thinking. The functional RG provides a non-perturbative approach that interpolates between weak and strong coupling.

The scale hierarchy (Step 1) compares the bandwidth $W \sim zt$ to the interaction strength $U$. Coarse-graining (Step 2) can proceed from weak coupling (integrating out high-energy particle-hole excitations) or strong coupling (integrating out doubly occupied states). Theory space (Step 3) includes the ratio $U/t$, the filling, and temperature. The beta functions (Step 4) are computed via perturbative or functional methods. Fixed-point analysis (Step 5) reveals multiple competing phases: Fermi liquid, Mott insulator, superconductor, and possibly strange metal.

\marginnote{The Hubbard model was introduced independently by Hubbard, Gutzwiller, and Kanamori in 1963 to explain magnetism and metal-insulator transitions.}

%-------------------------------------------------------------------------------
\section{The Hubbard Hamiltonian}
\label{sec:hubbard_hamiltonian}
%-------------------------------------------------------------------------------

The Hubbard model describes electrons hopping on a lattice with an on-site repulsion:
\begin{equation}
H = -t \sum_{\langle i,j \rangle, \sigma} (c^\dagger_{i\sigma} c_{j\sigma} + \text{h.c.}) + U \sum_i n_{i\uparrow} n_{i\downarrow}
\label{eq:hubbard}
\end{equation}
where $c^\dagger_{i\sigma}$ creates an electron with spin $\sigma$ at site $i$, $n_{i\sigma} = c^\dagger_{i\sigma} c_{i\sigma}$ is the number operator, $t$ is the hopping amplitude, and $U > 0$ is the on-site Coulomb repulsion.

\subsection{Scale Identification}

Following Step 1 of the recipe, we identify the scales. The bandwidth $W \sim zt$ (where $z$ is the coordination number) sets the kinetic energy scale, characterizing how much energy an electron gains by delocalizing across the lattice. The interaction $U$ is the potential energy scale, the cost of putting two electrons on the same site. The dimensionless coupling $u = U/W$ measures the relative strength of interactions: when $u \ll 1$ kinetic energy dominates and electrons form a Fermi liquid, while when $u \gg 1$ interactions dominate and electrons localize into a Mott insulator.

The scale parameter can be taken as the energy cutoff $\Lambda$ or temperature $T$. As we lower $\Lambda$ or $T$, the effective coupling can flow to strong or weak values depending on the bare parameters and dimensionality.

\marginnote{The ratio $U/t$ controls whether electrons behave as itinerant metals or localized moments.}

%-------------------------------------------------------------------------------
\section{Limiting Cases}
\label{sec:hubbard_limits}
%-------------------------------------------------------------------------------

The physics depends dramatically on the ratio $U/t$ and the electron filling.

\subsection{Weak Coupling: $U \ll t$}

For small $U$, electrons form a Fermi liquid with renormalized parameters. Standard perturbation theory in $U/t$ applies, and the system is metallic.

\subsection{Strong Coupling: $U \gg t$}

For large $U$, double occupancy is energetically suppressed. At half-filling (one electron per site on average), the system becomes a Mott insulator with localized spins. The effective low-energy theory is the Heisenberg antiferromagnet:
\begin{equation}
H_{\text{eff}} = J \sum_{\langle i,j \rangle} \mathbf{S}_i \cdot \mathbf{S}_j
\label{eq:heisenberg}
\end{equation}
with exchange coupling $J = 4t^2/U$.

\subsection{The Mott Transition}

As $U/t$ increases from zero, the system undergoes a metal-insulator transition (Mott transition) at some critical value $(U/t)_c$. This is a quantum phase transition driven by the competition between kinetic and potential energy.

%-------------------------------------------------------------------------------
\section{RG for the Hubbard Model}
\label{sec:hubbard_rg}
%-------------------------------------------------------------------------------

Several RG approaches have been developed for the Hubbard model.

\subsection{Weak Coupling RG}

For $U \ll t$, perturbative RG can be developed around the free electron fixed point. The beta function for the interaction strength depends on the dimensionality and band structure.

In one dimension with a linear spectrum near the Fermi points, the RG equations for the coupling constants $g_i$ (characterizing different scattering processes) take the form:
\begin{align}
\frac{dg_1}{d\ell} &= -g_1^2 + \cdots, \label{eq:hubbard_g1}\\
\frac{dg_2}{d\ell} &= -2g_1 g_2 + \cdots, \label{eq:hubbard_g2}
\end{align}
where $\ell = \ln(\Lambda_0/\Lambda)$ is the logarithmic scale.

\marginnote{The 1D Hubbard model is exactly solvable by the Bethe ansatz, providing a benchmark for RG calculations.}

\subsection{Strong Coupling RG}

For $U \gg t$, one starts from the atomic limit and treats hopping as a perturbation. The RG generates effective interactions at lower energy scales.

The key insight is that high-energy virtual excitations (creating double occupancies) are integrated out, generating the superexchange coupling $J$ and other effective interactions.

%-------------------------------------------------------------------------------
\section{Fixed Points and Phase Diagram}
\label{sec:hubbard_fixed}
%-------------------------------------------------------------------------------

Applying the framework of Chapter~\ref{ch:fixed_points}:

\subsection{One Dimension}

In 1D at half-filling, the repulsive Hubbard model flows toward a Mott insulating fixed point for any $U > 0$. The fixed point is described by a Luttinger liquid with gapless spin excitations governed by the effective Heisenberg antiferromagnetic coupling, and gapped charge excitations reflecting the Mott gap that prevents charge transport. Away from half-filling, the system remains metallic but as a Luttinger liquid rather than a Fermi liquid, with power-law correlations rather than quasiparticle excitations.

\subsection{Two Dimensions}

The 2D Hubbard model is believed to describe high-temperature superconductivity in the cuprates. The RG analysis reveals a complex phase diagram with multiple competing phases. At half-filling and sufficiently strong coupling $U > U_c$, the system forms an antiferromagnetic Mott insulator. Upon doping away from half-filling, d-wave superconductivity may emerge from the pairing of electrons mediated by antiferromagnetic fluctuations. At weak coupling, the system exhibits Fermi liquid behavior with well-defined quasiparticles.

The phase diagram involves multiple competing fixed points, and no single analytical method captures all regimes. Numerical methods including quantum Monte Carlo, dynamical mean-field theory, and tensor networks complement the analytical RG, each providing insight into different corners of parameter space.

\marginnote{The 2D Hubbard model remains one of the great unsolved problems in condensed matter physics.}

\subsection{Infinite Dimensions}

In the limit $d \to \infty$ (with proper scaling of $t$), the Hubbard model becomes exactly solvable via dynamical mean-field theory (DMFT). This provides a nonperturbative benchmark for understanding the Mott transition. A sharp first-order Mott transition occurs at $(U/W)_c \approx 1.5$, with a metal-insulator coexistence region where both phases are locally stable. The crossover from itinerant to localized behavior can be traced continuously as $U/W$ increases, with spectral weight transferring from coherent quasiparticle peaks to incoherent Hubbard bands.

%-------------------------------------------------------------------------------
\section{Functional RG Approach}
\label{sec:hubbard_frg}
%-------------------------------------------------------------------------------

The functional renormalization group provides a systematic nonperturbative framework for the Hubbard model. This approach, based on an exact flow equation for the effective action, allows interpolation between weak and strong coupling.

\subsection{The Effective Action}

Define the generating functional for connected Green's functions $W[J]$ and its Legendre transform, the effective action $\Gamma[\phi]$. Introducing a regulator $R_\Lambda$ that suppresses modes below the scale $\Lambda$:
\begin{equation}
\Gamma_\Lambda[\phi] = \sup_J \left( J \cdot \phi - W_\Lambda[J] \right) - \Delta S_\Lambda[\phi]
\end{equation}
where $\Delta S_\Lambda$ is the regulator contribution.

\subsection{The Flow Equation}

The exact RG equation for $\Gamma_\Lambda$ is:
\begin{equation}
\frac{\partial \Gamma_\Lambda}{\partial \Lambda} = \frac{1}{2} \text{Tr} \left[ \left( \Gamma_\Lambda^{(2)} + R_\Lambda \right)^{-1} \frac{\partial R_\Lambda}{\partial \Lambda} \right]
\label{eq:wetterich_hubbard}
\end{equation}
where $\Gamma_\Lambda^{(2)}$ is the second functional derivative of the effective action.

\marginnote{The functional RG provides a nonperturbative framework valid at any coupling strength.}

\subsection{Truncation and Results}

Practical calculations require truncating the effective action to a finite number of coupling constants. Common truncations include the static four-point vertex, which captures magnetic and pairing instabilities; the frequency-dependent vertex, which captures dynamic correlations; and self-energy effects, which capture spectral weight transfer between coherent and incoherent excitations. More sophisticated truncations include momentum dependence and higher-order vertices.

The functional RG has successfully predicted several key features of the Hubbard model phase diagram. Antiferromagnetic order emerges at half-filling when the nesting of the Fermi surface enhances magnetic susceptibility. Upon doping, d-wave pairing can arise from antiferromagnetic fluctuations acting as a pairing glue. Pseudogap behavior in the underdoped regime, where spectral weight is suppressed near the Fermi level even above the superconducting transition, also emerges naturally from the functional RG flow.

%-------------------------------------------------------------------------------
\section{Connection to the Anharmonic Oscillator}
\label{sec:hubbard_anharmonic}
%-------------------------------------------------------------------------------

There is a deep connection between the Hubbard model and the anharmonic oscillator of Chapter~\ref{ch:scale}.

\subsection{Path Integral Representation}

In the coherent state path integral, the Hubbard model becomes:
\begin{equation}
Z = \int \mathcal{D}[\bar{c}, c] \, e^{-S[\bar{c}, c]}
\end{equation}
with action containing quadratic (kinetic) and quartic (interaction) terms.

This has the same structure as the $\phi^4$ theory, with fermionic (Grassmann) fields instead of bosonic ones. The RG analysis proceeds similarly, with complications from the Fermi surface geometry.

\subsection{Local vs. Itinerant Physics}

The competition between $t$ (promoting delocalization) and $U$ (promoting localization) mirrors the competition between kinetic and potential energy in the anharmonic oscillator. The RG identifies which physics dominates at low energies.

%-------------------------------------------------------------------------------
\section{Emergent Phenomena}
\label{sec:hubbard_emergent}
%-------------------------------------------------------------------------------

The Hubbard model exhibits emergent phenomena that arise from the RG flow.

\subsection{Antiferromagnetism}

At half-filling with moderate $U$, antiferromagnetic order emerges below a N\'{e}el temperature $T_N$. The RG shows that antiferromagnetic fluctuations are relevant perturbations that flow to strong coupling.

\subsection{Superconductivity}

Upon doping, the antiferromagnetic fluctuations can mediate an effective attractive interaction between electrons, potentially leading to superconductivity. The RG identifies the pairing symmetry (typically d-wave in 2D cuprates).

\marginnote{The mechanism of high-temperature superconductivity remains one of the outstanding problems in physics.}

\subsection{Strange Metal}

In certain parameter regimes, the Hubbard model may flow toward a ``strange metal'' fixed point characterized by non-Fermi liquid behavior: linear-in-$T$ resistivity, anomalous scaling of transport properties, and absence of well-defined quasiparticles.

%-------------------------------------------------------------------------------
\section{Connection to the Geometric Framework}
\label{sec:hubbard_geometry}
%-------------------------------------------------------------------------------

We now connect the Hubbard model to the geometric framework of Part I.

\subsection{Theory Space}

The theory space for the Hubbard model includes the interaction strength $U/t$, the electron filling $n$ (from empty to half-filled to fully occupied), temperature $T$ or equivalently the energy scale $\Lambda$, and in extended models, longer-range interactions and additional orbitals. The RG flow traces a trajectory through this high-dimensional space, with different regions flowing to different fixed points.

\subsection{Fixed Points and Phases}

Different phases correspond to different fixed points in theory space. The Fermi liquid is described by a free fermion fixed point with renormalized parameters including the quasiparticle mass and Landau interaction parameters. The Mott insulator corresponds to a strong coupling fixed point with localized spins and a charge gap. The superconductor is a BCS-type fixed point with Cooper pairing. A strange metal phase, if it exists, would correspond to a non-Fermi liquid fixed point with anomalous scaling and no well-defined quasiparticles. Phase transitions are crossovers between basins of attraction of these different fixed points.

\subsection{Stability Analysis}

At each fixed point, the stability matrix of Chapter~\ref{ch:fixed_points} determines the fate of perturbations. Relevant perturbations destabilize the phase and drive transitions to other fixed points. Irrelevant perturbations flow to zero and do not affect the long-distance physics. Marginal perturbations require higher-order analysis to determine their ultimate behavior.

For the Fermi liquid fixed point in dimensions $d > 1$, forward scattering is marginal, corresponding to the renormalization of Landau parameters. BCS pairing is marginally relevant at zero temperature in the presence of an attractive interaction, explaining why arbitrarily weak attraction leads to superconductivity.

\subsection{The Metric on Theory Space}

While less developed than in CFT, a metric on the coupling space can be defined from susceptibilities:
\begin{equation}
G_{ij} = \frac{\partial^2 F}{\partial g^i \partial g^j}
\end{equation}
where $F$ is the free energy density and $g^i$ are the couplings.

This metric diverges at phase transitions (critical points), reflecting the singular behavior of the thermodynamic potentials.

%-------------------------------------------------------------------------------
\section*{Exercises}
\addcontentsline{toc}{section}{Exercises}
%-------------------------------------------------------------------------------

\begin{enumerate}
\item \textbf{Strong coupling expansion.} In the limit $U \gg t$, derive the effective Heisenberg Hamiltonian~\eqref{eq:heisenberg} using second-order perturbation theory in $t/U$.
\begin{enumerate}
\item Show that the superexchange coupling is $J = 4t^2/U$.
\item Explain why the effective interaction is antiferromagnetic ($J > 0$).
\item Discuss what happens at third order in $t/U$.
\end{enumerate}

\item \textbf{Fermi liquid stability.} For the 2D Hubbard model at weak coupling, the system is a Fermi liquid.
\begin{enumerate}
\item What is the quasiparticle residue $Z$ and how does it depend on $U/t$?
\item At what coupling strength do you expect Fermi liquid theory to break down?
\item Discuss the role of nesting in destabilizing the Fermi liquid.
\end{enumerate}

\item \textbf{Mott transition.} At half-filling, the Hubbard model undergoes a metal-insulator transition.
\begin{enumerate}
\item In the Brinkman-Rice picture, how does the quasiparticle weight $Z$ vanish as $U \to U_c$?
\item In DMFT, the transition can be first-order. Sketch the phase diagram in the $(U/t, T)$ plane.
\item Discuss the critical exponents associated with the Mott transition in mean-field theory.
\end{enumerate}

\item \textbf{Functional RG.} Consider the Wetterich equation~\eqref{eq:wetterich_hubbard} for the Hubbard model.
\begin{enumerate}
\item Explain the role of the regulator $R_\Lambda$.
\item What is the initial condition for $\Gamma_\Lambda$ at $\Lambda = \Lambda_0$?
\item How does the effective four-fermion interaction flow under the RG?
\end{enumerate}

\item \textbf{(Challenge) d-wave superconductivity.} In the 2D Hubbard model near half-filling:
\begin{enumerate}
\item Explain how antiferromagnetic fluctuations can mediate an effective attractive interaction.
\item Why is d-wave pairing favored over s-wave?
\item Relate the superconducting $T_c$ to the antiferromagnetic exchange coupling $J$.
\end{enumerate}
\end{enumerate}

%-------------------------------------------------------------------------------
\section{Summary}
\label{sec:hubbard_summary}
%-------------------------------------------------------------------------------

The Hubbard model demonstrates the six-step RG recipe in strongly correlated systems where perturbation theory is insufficient. The scale hierarchy (Step 1) compares kinetic energy $t$ to interaction $U$. Perturbative analysis (Step 2) reveals Gevrey-1 structure in both weak and strong coupling expansions. Theory space (Step 3) includes $U/t$, filling, and temperature, extended to include non-perturbative sectors. The beta functions (Step 4) are computed via weak-coupling RG, strong-coupling expansion, or the Wetterich equation~\eqref{eq:wetterich_hubbard}. Fixed-point analysis (Step 5) reveals competing phases: Fermi liquid, Mott insulator, superconductor, and possibly strange metal. Physical predictions require proper handling of non-perturbative effects (Step 6).

The competition between kinetic and potential energy scales creates a rich phase diagram with the metal-insulator (Mott) transition as its central feature. Different phases correspond to different fixed points in theory space, with phase transitions representing crossovers between basins of attraction. Emergent phenomena including magnetism and superconductivity arise naturally from the RG flow, even when they are not visible in the microscopic Hamiltonian.

The Hubbard model demonstrates that the geometric RG framework extends to strongly correlated quantum systems, providing organizing principles even when perturbative calculations fail. The functional RG offers non-perturbative access to the full phase diagram, interpolating between weak and strong coupling limits. The same six-step recipe that organized the anharmonic oscillator applies here, though the implementation requires more sophisticated machinery.

