%===============================================================================
\chapter{Perturbation Theory and UV Divergences}
\label{ch:perturbation}
%===============================================================================

\marginnote{Part I developed the exact RG framework. This chapter introduces perturbation theory as the primary computational method, covering both the universal divergence structure and the regularization/renormalization machinery needed for quantum field theory.}

The RG framework developed in Part I is \textbf{exact}---beta functions, fixed points, and flows exist independently of how we compute them. \textbf{Perturbation theory} is the most common method for computing these quantities: expand in a small parameter and calculate order by order.

This chapter examines perturbation theory comprehensively:
\begin{itemize}
\item \textbf{Section~\ref{sec:why_diverge}}: Why perturbation series generically diverge
\item \textbf{Section~\ref{sec:pert_examples}}: Three canonical examples demonstrating universality
\item \textbf{Section~\ref{sec:decision_tree}}: When RG methods are needed versus simpler approaches
\item \textbf{Section~\ref{sec:sethna_template}}: A systematic problem-solving methodology
\item \textbf{Section~\ref{sec:regularization}}: UV divergences and regularization methods
\item \textbf{Section~\ref{sec:renorm_schemes}}: Renormalization schemes and their equivalence
\end{itemize}

The key insight is that perturbation theory, while powerful, is \emph{incomplete}. The factorial divergence of perturbative series encodes information about non-perturbative physics---a theme we develop fully in Chapter~\ref{ch:resurgence}.

%-------------------------------------------------------------------------------
\section{Why Perturbation Series Diverge}
\label{sec:why_diverge}
%-------------------------------------------------------------------------------

Before developing the machinery, let's understand \emph{why} perturbation series in physics generically diverge.

\subsection{The Source of Factorial Growth}

Consider a generic nonlinear problem with small parameter $\epsilon$:
\begin{equation}
\mathcal{L}[f] = \epsilon \mathcal{N}[f]
\end{equation}
where $\mathcal{L}$ is linear and $\mathcal{N}$ is nonlinear. The perturbative solution $f = \sum_n \epsilon^n f_n$ is constructed iteratively:
\begin{equation}
f_{n+1} = \mathcal{L}^{-1}[\mathcal{N}[f_0 + \epsilon f_1 + \cdots + \epsilon^n f_n]]
\end{equation}

\marginnote{Each order of perturbation theory involves applying the nonlinearity to all previous orders. This generates combinatorial factors.}

At order $n$, we must account for all ways of distributing $n$ powers of $\epsilon$ among the nonlinear terms. The number of such distributions grows combinatorially. For a cubic nonlinearity, the growth is roughly $n!$.

\textbf{Dyson's argument:} For quantum field theories, Dyson argued that the perturbative series must diverge. If the series converged for coupling $g > 0$, it would converge in a disk including $g < 0$. But for $g < 0$, the vacuum is unstable (the potential is unbounded below), so the theory doesn't exist. Hence convergence is impossible.

\subsection{Gevrey-1 Structure}

\marginnote{Gevrey-1 means factorial growth: $|a_n| \lesssim n!$. This is the generic case for physical perturbation series.}

A formal series $\tilde{f}(\epsilon) = \sum_{n=0}^\infty a_n \epsilon^n$ is \textbf{Gevrey of order 1} (Gevrey-1) if:
\begin{equation}
|a_n| \leq C \cdot K^n \cdot n!
\label{eq:gevrey1_def}
\end{equation}
for constants $C, K > 0$. The factorial $n!$ means the series has zero radius of convergence.

\textbf{Physical examples:}
\begin{itemize}
\item The anharmonic oscillator ground state energy has $a_n \sim (-1)^n \cdot \text{const} \cdot A^n \cdot n!$
\item QED perturbation theory has $a_n \sim n! \cdot (1/137)^n$ from diagram counting
\item The epsilon expansion for critical exponents has factorially growing coefficients from renormalon contributions
\item The late-time behavior of the Lorenz system near bifurcation has factorially divergent corrections
\item Matched asymptotic expansions in fluid mechanics (boundary layers, etc.) generically produce Gevrey-1 series
\end{itemize}

\marginnote{Divergent series are universal: they appear in ODEs, PDEs, and QFT alike. The mathematical structure is independent of the physical origin.}

The key observation is that factorial divergence is \textbf{not specific to quantum field theory}. It appears whenever:
\begin{enumerate}
\item A nonlinearity generates combinatorial complexity at each order
\item A small parameter controls the expansion
\item The expansion is around a singular limit (e.g., $\epsilon \to 0$ in the anharmonic oscillator)
\end{enumerate}

\subsection{What Divergence Encodes}

The crucial insight is that factorial divergence is not random. The \emph{pattern} of divergence---signs, growth rates, subleading corrections---encodes non-perturbative physics that is invisible to any finite truncation of the series.

\marginnote{The way a series diverges tells you about physics invisible to any finite truncation. Chapter~\ref{ch:resurgence} develops the tools to extract this information.}

This is a profound observation: the perturbative series ``knows'' about non-perturbative effects like tunneling and instantons, even though these effects are exponentially suppressed and invisible at any finite order. Chapter~\ref{ch:resurgence} develops the machinery---Borel transforms, transseries, and alien calculus---to systematically extract this hidden information.

\subsection{Divergent Series in Classical Mechanics and PDEs}

\marginnote{Divergent series are not a quantum phenomenon. They appear throughout classical mechanics, fluid dynamics, and nonlinear PDEs.}

It is essential to emphasize that factorial divergence is \textbf{not unique to quantum mechanics or field theory}. Classical dynamical systems exhibit the same structure:

\textbf{The Lorenz system:} Near the Hopf bifurcation at $\rho = 1$, perturbative corrections to the fixed point position diverge factorially. The pattern of divergence encodes information about the global structure of the unstable manifold.

\textbf{Boundary layer theory:} The Prandtl matched asymptotic expansion for boundary layers in fluid mechanics produces Gevrey-1 series. The divergence encodes the ``inner'' scale physics invisible to the ``outer'' expansion.

\textbf{The porous medium equation:} Perturbative corrections to the Barenblatt self-similar solution (expanding around $m = 1$) diverge factorially for $m \neq 1$. This reflects the singular nature of the nonlinear diffusion.

\textbf{Singular perturbation theory:} Any problem of the form $\epsilon \mathcal{L}_1[f] + \mathcal{L}_0[f] = 0$ with $\epsilon \to 0$ generically produces factorially divergent series. The boundary layer, turning point, and WKB analyses of asymptotic methods are all Gevrey-1.

\begin{workedbox}[Box 5.1: Divergent Series in the Van der Pol Oscillator]
\textbf{The model:} The Van der Pol equation $\ddot{x} + \epsilon(x^2 - 1)\dot{x} + x = 0$ with $\epsilon \ll 1$ describes a weakly nonlinear oscillator.

\textbf{The expansion:} The limit cycle amplitude can be expanded:
\begin{equation}
A(\epsilon) = 2 + a_1 \epsilon + a_2 \epsilon^2 + a_3 \epsilon^3 + \cdots
\end{equation}

\textbf{The divergence:} The coefficients grow as $a_n \sim n!$ for large $n$. This is because each order of perturbation theory involves iterating the nonlinearity, generating combinatorial growth.

\textbf{The physics:} The divergence reflects the \emph{relaxation oscillation} regime at large $\epsilon$. Information about this strong-coupling behavior is encoded in how the weak-coupling series diverges.

\textbf{Comparison with QFT:} The mathematical structure---Gevrey-1 divergence, Borel summability, Stokes phenomena---is identical to QFT perturbation theory. The techniques of Chapter~\ref{ch:resurgence} apply without modification.
\end{workedbox}

This universality is why we develop the resurgent framework in generality: the tools work for ODEs, PDEs, and QFT alike.

%-------------------------------------------------------------------------------
\section{Perturbation Theory for Nonlinear PDEs}
\label{sec:pde_perturbation}
%-------------------------------------------------------------------------------

\marginnote{Nonlinear PDEs provide concrete examples where perturbation theory breaks down in familiar ways. The same RG methods that work in QFT apply directly to these classical equations.}

Before examining the canonical quantum and field-theoretic examples, we develop perturbation theory for nonlinear partial differential equations. This serves several pedagogical purposes. First, PDEs are more familiar than quantum field theories, providing concrete physical intuition. Second, many PDE problems can be solved exactly or numerically, allowing us to verify perturbative predictions. Third, the breakdown of naive perturbation theory in PDEs exhibits the same universal patterns as in QFT, demonstrating that secular terms, small denominators, and the need for renormalization are not quantum phenomena but general features of perturbative expansions.

The strategy parallels what Barenblatt called intermediate asymptotics (Chapter~\ref{ch:rg_geometry}). We seek solutions valid for intermediate times or length scales where details of initial conditions have been forgotten but the system has not yet reached equilibrium. Naive perturbation theory fails in this regime because small denominators or secular terms accumulate. The renormalization group provides systematic resummation that extends the perturbative solution to the intermediate asymptotic regime.

\subsection{Why Perturbation Theory Fails for Nonlinear PDEs}

Consider a generic nonlinear PDE of the form
\begin{equation}
\frac{\partial u}{\partial t} = L[u] + \epsilon N[u]
\end{equation}
where $L$ is a linear operator and $N$ is nonlinear. The perturbative solution is
\begin{equation}
u(x,t) = u_0(x,t) + \epsilon u_1(x,t) + \epsilon^2 u_2(x,t) + \cdots
\end{equation}

At zeroth order, $\partial_t u_0 = L[u_0]$ is linear and typically solvable. At first order, $\partial_t u_1 = L[u_1] + N[u_0]$ is an inhomogeneous linear equation with forcing term $N[u_0]$. The key observation is that if $N[u_0]$ resonates with eigenmodes of $L$, the solution $u_1$ will contain \textbf{secular terms} that grow unboundedly in time or space. These secular terms invalidate the expansion for $t$ or $|x|$ large compared to $1/\epsilon$.

The physical origin is clear. The nonlinearity drives the system at frequencies or wavelengths matching the natural modes of the linear operator. This resonant forcing produces a response that accumulates over time. The perturbative expansion assumes $\epsilon u_1 \ll u_0$, but this breaks down when secular terms make $u_1 \sim t \cdot u_0$ for $t \gtrsim 1/\epsilon$.

The renormalization group resolves this breakdown by absorbing secular terms into time-dependent (or scale-dependent) parameters. Instead of fixed constants appearing in $u_0$, we allow them to run with $t$ or $|x|$ according to RG equations. The running is chosen precisely to cancel the secular growth at each order. This produces a uniformly valid expansion for all times or length scales.

\begin{workedbox}[Box 5.X: Nonlinear Heat Equation with Secular Terms]
\textbf{Goal:} Demonstrate how secular terms arise in a simple nonlinear PDE and how RG methods provide systematic resummation. This example serves as a template for more complex problems.

\textbf{Setup:} Consider the nonlinear heat equation with a quadratic nonlinearity:
\begin{equation}
\frac{\partial u}{\partial t} = \frac{\partial^2 u}{\partial x^2} + \epsilon u^2
\label{eq:nonlinear_heat}
\end{equation}
on the infinite line $-\infty < x < \infty$ with initial condition $u(x,0) = f(x)$ where $f(x)$ is localized (decays as $|x| \to \infty$).

\textbf{Step 1: Zeroth-order solution.}

At $\epsilon = 0$, we have the linear heat equation $\partial_t u_0 = \partial_x^2 u_0$. The solution is
\begin{equation}
u_0(x,t) = \int_{-\infty}^\infty G(x-y,t)f(y)\,dy = \frac{1}{\sqrt{4\pi t}}\int_{-\infty}^\infty e^{-(x-y)^2/(4t)}f(y)\,dy
\end{equation}
where $G(x,t) = (4\pi t)^{-1/2}e^{-x^2/(4t)}$ is the heat kernel. For localized initial data with total "mass" $M = \int f(x)dx$, the long-time behavior is
\begin{equation}
u_0(x,t) \sim \frac{M}{\sqrt{4\pi t}}e^{-x^2/(4t)}, \quad t \gg 1
\end{equation}

\textbf{Step 2: First-order correction and secular terms.}

At first order in $\epsilon$, the equation is
\begin{equation}
\frac{\partial u_1}{\partial t} = \frac{\partial^2 u_1}{\partial x^2} + u_0^2
\end{equation}
The forcing term $u_0^2$ acts as a source. Since $u_0 \to 0$ as $t \to \infty$, we might expect $u_1$ to remain bounded. However, this is wrong. The integral
\begin{equation}
u_1(x,t) = \int_0^t d\tau \int_{-\infty}^\infty dy\, G(x-y,t-\tau)u_0(y,\tau)^2
\end{equation}
accumulates contributions from all earlier times $\tau < t$. Even though $u_0(\tau)^2$ decays for each fixed $\tau$, the time integral causes secular growth.

To see this explicitly, substitute the asymptotic form of $u_0$ for large times:
\begin{equation}
u_1(x,t) \sim \int_0^t d\tau \frac{M^2}{4\pi\tau}e^{-x^2/(2\tau)} = \frac{M^2}{4\pi}\int_0^t \frac{d\tau}{\tau}e^{-x^2/(2\tau)}
\end{equation}
Changing variables to $s = x^2/(2\tau)$ gives
\begin{equation}
u_1(x,t) \sim \frac{M^2}{4\pi}\int_{x^2/(2t)}^\infty \frac{ds}{s}e^{-s} \sim \frac{M^2}{4\pi}\log\left(\frac{2t}{x^2}\right), \quad t \gg x^2
\end{equation}
This logarithmic growth is a \textbf{secular term}. The first-order correction grows as $\log t$, violating the assumption that $\epsilon u_1 \ll u_0 \sim t^{-1/2}$ for $t \gtrsim 1/\epsilon^2$.

\textbf{Step 3: Origin of the secular term.}

The secular growth arises because $u_0^2$ sources the diffusion equation with a term that integrates over time. Physically, the nonlinear term continuously adds heat to the system. Even though the rate decreases as $u_0$ decays, the integrated effect accumulates logarithmically.

From the RG perspective, the problem is that we assumed fixed parameters in $u_0$. The correct approach is to let the "mass" $M$ run with time to absorb the accumulated effect of the nonlinearity.

\textbf{Step 4: RG improvement.}

Define a running mass $M(t)$ and write
\begin{equation}
u(x,t) = \frac{M(t)}{\sqrt{4\pi t}}e^{-x^2/(4t)} + O(\epsilon^2)
\end{equation}
Substituting into the full nonlinear equation~\eqref{eq:nonlinear_heat} and demanding that secular terms cancel at each order gives
\begin{equation}
\frac{dM}{dt} = \epsilon\int_{-\infty}^\infty u^2\,dx \approx \frac{\epsilon M^2}{4\pi t}
\end{equation}
Solving this RG equation:
\begin{equation}
M(t) = \frac{M(0)}{1 - \epsilon M(0)\log(t)/(4\pi)}
\end{equation}

The renormalized solution is
\begin{equation}
u_{\text{RG}}(x,t) = \frac{M(0)}{\sqrt{4\pi t}(1 - \epsilon M(0)\log t/(4\pi))}e^{-x^2/(4t)}
\end{equation}
This is uniformly valid for all $t$. The naive perturbative result corresponds to expanding the denominator, which reproduces $M(0) + \epsilon M(0)^2\log(t)/(4\pi)$, but the RG result correctly resums these logarithms.

\textbf{Step 5: Physical interpretation and comparison.}

The running mass $M(t)$ increases logarithmically due to the positive nonlinear term $+\epsilon u^2$. This represents a feedback where heat diffusion is enhanced by the nonlinearity. For $\epsilon < 0$ (a negative nonlinearity), $M(t)$ would decrease, potentially leading to finite-time blowup if $1 - \epsilon M(0)\log t/(4\pi) \to 0$.

Numerical solution of equation~\eqref{eq:nonlinear_heat} confirms that $u_{\text{RG}}$ captures the long-time behavior correctly. The naive perturbative result fails for $t \gg \exp(4\pi/|\epsilon M(0)|)$, while the RG result remains accurate.

\textbf{Key Insight:} This example demonstrates the core RG mechanism for PDEs. Secular terms signal that fixed parameters are incorrect. Running parameters absorb the secular growth. The RG equations determine how parameters evolve to maintain consistency. This is exactly the same structure as in QFT renormalization, but here in a completely classical setting with no quantum mechanics or field operators involved.
\end{workedbox}

\begin{workedbox}[Box 5.Y: Burgers Equation and Shock Formation]
\textbf{Goal:} Demonstrate RG methods for PDEs with more complex nonlinear structure. Burgers equation exhibits shock formation, and RG provides systematic description of the shock layer structure.

\textbf{Setup:} The Burgers equation is
\begin{equation}
\frac{\partial u}{\partial t} + u\frac{\partial u}{\partial x} = \nu\frac{\partial^2 u}{\partial x^2}
\label{eq:burgers}
\end{equation}
This models nonlinear wave propagation with diffusion. The nonlinear term $u\partial_x u$ causes wavefront steepening. The diffusion $\nu\partial_x^2 u$ opposes steepening. For small viscosity $\nu \ll 1$, shocks (discontinuities in $u$) can form.

\textbf{Step 1: Inviscid limit and shock formation.}

For $\nu = 0$, Burgers equation reduces to the inviscid Burgers equation $\partial_t u + u\partial_x u = 0$. This is solved by the method of characteristics. Characteristics are straight lines in the $(x,t)$ plane along which $u$ is constant. The slope of a characteristic starting at $(x_0,0)$ is $dx/dt = u(x_0,0)$.

If the initial profile $u(x,0)$ decreases anywhere (i.e., $\partial_x u(x,0) < 0$), characteristics with different speeds will intersect. At the intersection point, $u$ becomes multivalued, signaling shock formation. The shock time is
\begin{equation}
t_{\text{shock}} \sim \frac{1}{\max|\partial_x u(x,0)|}
\end{equation}

\textbf{Step 2: Regularization by diffusion.}

For $\nu > 0$, diffusion smooths the shock. Instead of a true discontinuity, we get a sharp transition layer of width $\delta \sim \sqrt{\nu t}$. Within this layer, $\partial_x u \sim U/\delta$ where $U$ is the jump in $u$ across the shock. Balancing the nonlinear and diffusion terms in equation~\eqref{eq:burgers}:
\begin{equation}
U\frac{U}{\delta} \sim \nu\frac{U}{\delta^2} \quad \Rightarrow \quad \delta \sim \frac{\nu}{U}
\end{equation}
This is the shock layer width.

\textbf{Step 3: Perturbative treatment for small viscosity.}

For $\nu \ll 1$, we seek a perturbative solution. The natural approach is to expand $u = u_0 + \nu u_1 + \nu^2 u_2 + \cdots$ where $u_0$ solves the inviscid equation. However, this fails near the shock. The derivative $\partial_x u_0$ becomes infinite at the shock, making $\nu\partial_x^2 u_0$ singular. The perturbative expansion breaks down in the shock layer.

The resolution is matched asymptotic expansions: construct an "outer" solution away from the shock valid for $\nu \to 0$, and an "inner" solution within the shock layer where diffusion is important. The RG provides a systematic way to implement this matching.

\textbf{Step 4: RG analysis of the shock layer.}

Define a shock position $x_s(t)$ and width $\delta(t)$. Write
\begin{equation}
u(x,t) = u_L + \frac{u_R - u_L}{2}\left[1 + \tanh\left(\frac{x - x_s(t)}{\delta(t)}\right)\right]
\end{equation}
where $u_L, u_R$ are the values to the left and right of the shock. This ansatz interpolates smoothly from $u_L$ to $u_R$ over a width $\delta$.

Substituting into Burgers equation~\eqref{eq:burgers} and matching coefficients gives RG equations for $x_s(t)$ and $\delta(t)$:
\begin{align}
\frac{dx_s}{dt} &= \frac{u_L + u_R}{2} \\
\frac{d\delta}{dt} &= \frac{\nu}{\delta} - \frac{(u_R - u_L)\delta}{12}
\end{align}
The first equation says the shock propagates at the average velocity. The second equation is the RG equation for the shock width. It balances diffusive broadening ($+\nu/\delta$) against nonlinear steepening ($-(u_R - u_L)\delta/12$).

At late times, $\delta$ approaches a quasi-steady state where $d\delta/dt \approx 0$:
\begin{equation}
\delta_{\text{steady}} \sim \sqrt{\frac{12\nu}{u_R - u_L}}
\end{equation}
This recovers the scaling $\delta \sim \sqrt{\nu/U}$ obtained from dimensional analysis.

\textbf{Step 5: Connection to turbulence and the Kolmogorov spectrum.}

Burgers equation is intimately connected to the theory of turbulence. Kolmogorov's theory of turbulence postulates that energy cascades from large scales to small scales where it is dissipated by viscosity. Burgers equation captures this cascade in one dimension.

The power spectrum of velocity fluctuations in Burgers turbulence has been computed numerically and analytically. For small viscosity, the spectrum exhibits a power law $E(k) \sim k^{-2}$ at intermediate wavenumbers. This contrasts with the Kolmogorov $k^{-5/3}$ spectrum in three-dimensional Navier-Stokes turbulence, but the conceptual structure is similar. The RG systematically organizes the cascade from large to small scales.

Polyakov developed an RG approach to Burgers turbulence in the 1990s, showing that the intermittency corrections (deviations from Kolmogorov scaling) can be computed systematically using RG methods. This work demonstrated that RG ideas, initially developed for equilibrium critical phenomena, apply to far-from-equilibrium nonlinear dynamics.

\textbf{Key Insight:} Burgers equation demonstrates that RG methods handle shock formation and multi-scale structure in nonlinear PDEs. The shock layer is an example of a "boundary layer" requiring matched asymptotic expansions. The RG provides systematic machinery for constructing these expansions and ensuring their consistency. The same methods apply to more complex fluid mechanics problems including Navier-Stokes turbulence, boundary layers in aerodynamics, and nonlinear wave propagation.
\end{workedbox}

\subsection{The $\epsilon$-Expansion for Anomalous Dimensions in PDEs}

The porous medium equation with water retention (Chapter~\ref{ch:rg_geometry}, Box~2.4) provides a paradigmatic example of systematic computation of anomalous dimensions using $\epsilon$-expansion. The method directly parallels the $d = 4 - \epsilon$ expansion in $\phi^4$ theory.

\textbf{Recap of the Problem:} The modified Boussinesq equation for groundwater spreading is
\begin{equation}
\frac{\partial p}{\partial t} = \begin{cases}
\kappa \nabla \cdot (p\nabla p), & \partial p/\partial t \geq 0 \\
\kappa_1 \nabla \cdot (p\nabla p), & \partial p/\partial t < 0
\end{cases}
\end{equation}
where $\kappa_1 > \kappa$ due to capillary retention. Define $\epsilon = \kappa_1/\kappa - 1 > 0$.

\textbf{The Zeroth-Order Solution ($\epsilon = 0$):} Complete similarity gives
\begin{equation}
p_0(r,t) = \frac{Q^{1/2}}{\kappa^{1/2}t^{1/2}}\Phi_0\left(\frac{r}{(Q\kappa t)^{1/4}}\right), \quad r_f(t) = A_0(Q\kappa t)^{1/4}
\end{equation}
where $A_0$ and $\Phi_0$ are determined by solving an ODE. This gives the exponent $\beta_0 = 1/4$.

\textbf{First-Order Correction ($\epsilon^1$):} Assume
\begin{equation}
\beta = \beta_0 + \epsilon\beta_1 + \epsilon^2\beta_2 + \cdots = \frac{1}{4} + \epsilon\beta_1 + O(\epsilon^2)
\end{equation}
Expanding the self-similar profile as $\Phi = \Phi_0 + \epsilon\Phi_1 + \cdots$ and substituting into the modified equation gives a linear ODE for $\Phi_1$. The solvability condition (requiring $\Phi_1$ to vanish at the boundary with correct asymptotics) determines $\beta_1$.

Chen and Goldenfeld computed this explicitly using RG methods, finding
\begin{equation}
\beta_1 = 0, \quad \beta_2 = -\frac{1}{16}
\end{equation}
so that
\begin{equation}
\beta(\epsilon) = \frac{1}{4} - \frac{\epsilon^2}{16} + O(\epsilon^3)
\end{equation}
This agrees with numerical integration and rigorous asymptotic analysis.

\textbf{Systematic Procedure:} The $\epsilon$-expansion follows a standard recipe:
\begin{enumerate}
\item Identify the zeroth-order solution (complete similarity, no anomalous dimension)
\item Expand exponents and profiles in powers of $\epsilon$
\item Substitute into the PDE and collect terms at each order in $\epsilon$
\item Solve the resulting hierarchy of linear problems
\item Impose solvability conditions that fix the corrections to anomalous dimensions
\end{enumerate}

This procedure is identical in structure to the loop expansion in QFT. The "loops" here are successive orders in $\epsilon$. Each order involves solving an inhomogeneous linear problem whose source comes from lower orders. Secular terms at order $n$ fix the anomalous dimension correction at order $n$.

\textbf{Connection to Wilson-Fisher Fixed Point:} The $\epsilon$-expansion for critical exponents in $\phi^4$ theory uses dimensional continuation $d = 4 - \epsilon$. At $\epsilon = 0$ (four dimensions), the theory is at the Gaussian fixed point with no anomalous dimensions. For $\epsilon > 0$, the Wilson-Fisher fixed point appears with anomalous dimensions $\eta(\epsilon), \nu(\epsilon)$ computed order by order in $\epsilon$.

The porous medium $\epsilon$-expansion is completely analogous. At $\epsilon = 0$ ($\kappa_1 = \kappa$), we have complete similarity with $\beta = 1/4$ exactly. For $\epsilon > 0$, incomplete similarity appears with anomalous dimension $\beta(\epsilon)$ that must be computed perturbatively. The mathematics is identical; only the physics differs.

%-------------------------------------------------------------------------------
\section{Three Canonical Examples}
\label{sec:pert_examples}
%-------------------------------------------------------------------------------

To ground the abstract discussion, we now examine three canonical examples that demonstrate the universality of perturbative structure across different physical domains. These examples form a ladder of increasing complexity.

\marginnote{The three examples form a ladder: oscillator $\to$ field theory $\to$ PDE. Each adds capabilities the previous lacked.}

\subsection{The Anharmonic Oscillator}

The anharmonic oscillator is the simplest example and suffices to demonstrate secular terms and running parameters, the resolution via RG equations, Gevrey-1 divergence and the Borel plane, and the basic transseries structure.

It is too simple for non-trivial fixed points, operator mixing or anomalous dimensions, and statistical RG with coarse-graining.

\begin{workedbox}[Box 5.2: Complete Analysis of the Damped Anharmonic Oscillator]
\textbf{Scales and divergence.}
UV scale: oscillation period $\tau_{\text{fast}} \sim 1/\omega_0$.
IR scale: amplitude-decay time $\tau_{\text{slow}} \sim 1/\gamma$.
Small parameters: $\gamma \ll \omega_0$ (weak damping), $\epsilon \ll 1$ (weak nonlinearity).
Breakdown: secular terms at $t \sim \tau_{\text{slow}}$.
Non-perturbative: complex-time instantons.

\textbf{Perturbation theory.}
The perturbative solution $x(t) = A\cos(\omega_0 t) + O(\epsilon)$ develops secular terms.
The frequency series $\omega = \omega_0(1 + \frac{3\epsilon A^2}{8\omega_0^2} + c_2\epsilon^2 + \cdots)$ diverges with $|c_n| \sim n!$.
The Borel transform $\hat{\omega}(\zeta)$ has singularities at $\zeta = \omega_0^3/(3\epsilon)$ (instanton action).

\textbf{Running parameters.}
Perturbative: $(A, \phi)$.
Extended: $(A, \phi, \sigma)$ with $\sigma$ weighting instanton sector.

\textbf{Beta functions.}
\begin{align}
\frac{dA}{dt} &= -\gamma A \\
\frac{d\phi}{dt} &= \frac{3\epsilon A^2}{8\omega_0}
\end{align}
Transseries corrections: $O(\sigma e^{-S/\epsilon})$.
The Stokes constant $S_1$ relates perturbative and instanton sectors.

\textbf{Fixed points and stability.}
Perturbative fixed point: $A = 0$ (trivial, stable due to damping).
No non-perturbative fixed points.
All $A > 0$ trajectories flow to $A = 0$.

\textbf{Physical prediction.}
The effective frequency is:
\begin{equation}
\omega_{\text{eff}} = \omega_0\left(1 + \frac{3\epsilon A^2}{8\omega_0^2}\right) + O(\epsilon^2)
\end{equation}
For quantitative accuracy at larger $\epsilon$, resum using median prescription.
\end{workedbox}

\marginnote{The damped anharmonic oscillator example is developed fully in Chapter~\ref{ch:resurgence}, where we show how to extract non-perturbative physics from the factorial divergence.}

\subsection{The 1D $\phi^4$ Theory}

The 1D $\phi^4$ theory adds non-trivial beta functions with multiple couplings, the Gaussian fixed point and its stability, renormalon singularities from RG running, and statistical mechanics interpretation.

It is still too simple for non-trivial interacting fixed points (which require $d < 4$) and anomalous dimensions.

\begin{workedbox}[Box 5.3: Complete Analysis of 1D $\phi^4$ Theory]
\textbf{Scales and divergence.}
UV scale: cutoff $\Lambda$ (lattice spacing).
IR scale: correlation length $\xi \sim 1/\sqrt{r}$.
Small parameter: $\lambda/\Lambda^2 \ll 1$.
Breakdown: tadpole corrections grow with $\Lambda$.
Non-perturbative: renormalons from RG running.

\textbf{Perturbation theory.}
The beta functions $\beta_r = 2r + 3\lambda\Lambda/\pi(\Lambda^2 + r)$ and $\beta_\lambda = 2\lambda$ are perturbative leading terms.
Higher-order coefficients grow factorially.
The Borel transform has renormalon singularities at $\zeta_k = k/2$.

\textbf{Running parameters.}
Perturbative: $(r, \lambda)$.
Extended: $(r, \lambda, \sigma_{\text{ren}})$.

\textbf{Beta functions.}
\begin{align}
\beta_r &= 2r + \frac{3\lambda\Lambda}{\pi(\Lambda^2 + r)} + O(\sigma_{\text{ren}}e^{-1/(2\lambda)}) \\
\beta_\lambda &= 2\lambda + O(\sigma_{\text{ren}}e^{-1/(2\lambda)})
\end{align}
The renormalon Stokes constant: $S_{\text{ren}} = 1/\beta_1 + O(1) = 1/2 + O(1)$.

\textbf{Fixed points and stability.}
Perturbative: Gaussian fixed point $(0, 0)$.
Stability matrix eigenvalues: both $= 2$ (relevant, unstable).
No non-perturbative fixed points in 1D.
In $d = 4 - \epsilon$, the Wilson-Fisher fixed point appears.

\textbf{Physical prediction.}
Running couplings:
\begin{equation}
\lambda(\mu) = \lambda_0\left(\frac{\mu}{\mu_0}\right)^2
\end{equation}
Physical correlation functions computed from resummed expressions.
\end{workedbox}

\subsection{The Porous Medium Equation}

The porous medium equation adds anomalous dimensions (second-kind self-similarity), non-trivial scaling exponents from dynamics, Wasserstein gradient flow structure, and selection principles for physical solutions.

Together, the three examples demonstrate the complete framework. Any new problem will share features with one or more of these examples, and the techniques transfer accordingly.

\begin{workedbox}[Box 5.4: Complete Analysis of the Porous Medium Equation]
\textbf{Scales and divergence.}
UV scale: initial localization width.
IR scale: late-time spread $L(t) \sim t^\beta$.
Small parameter: $(m - 1)$ (deviation from linear diffusion).
Breakdown: anomalous exponent $\beta \neq 1/2$ for $m \neq 1$.
Non-perturbative: sub-leading self-similar modes.

\textbf{Perturbation theory.}
Expand $\beta(m)$ around $m = 1$:
\begin{equation}
\beta = \frac{1}{2} - \frac{d}{4}(m-1) + O((m-1)^2)
\end{equation}
This is asymptotic with singularities corresponding to competing modes.

\textbf{Running parameters.}
The exponent $\beta$ is determined by the self-similar ansatz.
Extended space includes mode weights selecting among solutions.

\textbf{Selection principle.}
Mass conservation: $\alpha = d\beta$.
Self-consistency: $\beta(md + 2 - d) = 1$.
Result:
\begin{equation}
\beta = \frac{1}{d(m-1) + 2}
\end{equation}
The physical mode is selected by boundary conditions (finite mass, compact support).

\textbf{Fixed points and stability.}
The Barenblatt profile is the unique stable self-similar attractor.
Other self-similar modes exist but are unstable.

\textbf{Physical prediction.}
The late-time density profile:
\begin{equation}
\rho(x, t) = \frac{1}{t^\alpha}\left[C - \frac{(m-1)}{4md}\frac{|x|^2}{(Dt)^{2\beta}}\right]_+^{1/(m-1)}
\end{equation}
This is exact for the PME. More general nonlinear diffusion would require resummation.
\end{workedbox}

%-------------------------------------------------------------------------------
\section{When Is RG Needed? A Decision Tree}
\label{sec:decision_tree}
%-------------------------------------------------------------------------------

Not every problem requires the full RG machinery. Following Sethna's pedagogical approach, we provide a decision tree for determining when RG methods are essential versus when simpler approaches suffice.

\marginnote{This decision tree helps identify whether full RG analysis is needed or if simpler methods suffice.}

\subsection{The Diagnostic Questions}

Ask the following questions in order:

\textbf{1. Is there a scale hierarchy?}
\begin{itemize}
\item If \textbf{NO}: Standard methods apply. Perturbation theory converges; no running parameters needed.
\item If \textbf{YES}: Proceed to question 2.
\end{itemize}

\textbf{2. Do naive methods exhibit pathologies?}

Look for secular terms (growing corrections), UV/IR divergences, or boundary layer mismatches.
\begin{itemize}
\item If \textbf{NO}: Scale separation is benign. Use matched asymptotics or multiple scales without full RG.
\item If \textbf{YES}: Running parameters are needed. Proceed to question 3.
\end{itemize}

\textbf{3. Are you near a phase transition or bifurcation?}
\begin{itemize}
\item If \textbf{NO}: Perturbative RG (few running parameters, truncated beta functions) may suffice.
\item If \textbf{YES}: Non-perturbative effects matter. Proceed to question 4.
\end{itemize}

\textbf{4. Are universal critical exponents or scaling functions needed?}
\begin{itemize}
\item If \textbf{NO}: Mean-field or Landau theory may be adequate.
\item If \textbf{YES}: Full RG analysis with fixed points, stability analysis, and possibly resummation is required.
\end{itemize}

\begin{workedbox}[Box 5.5: The Decision Tree Applied]
\textbf{Example 1: Simple harmonic oscillator}
\begin{itemize}
\item Scale hierarchy? NO (single timescale $1/\omega_0$)
\item $\Rightarrow$ No RG needed. Exact solution exists.
\end{itemize}

\textbf{Example 2: Damped anharmonic oscillator with $\epsilon \ll 1$, $\gamma \ll \omega_0$}
\begin{itemize}
\item Scale hierarchy? YES ($1/\omega_0$ vs $1/\gamma$ and $\omega_0/\epsilon A^2$)
\item Pathologies? YES (secular terms at $O(\epsilon t)$)
\item Near bifurcation? NO (far from any transition)
\item $\Rightarrow$ Perturbative RG (Lindstedt-Poincar\'e/multiple scales) suffices.
\end{itemize}

\textbf{Example 3: Ising model at $T \approx T_c$}
\begin{itemize}
\item Scale hierarchy? YES (lattice spacing $a$ vs correlation length $\xi \to \infty$)
\item Pathologies? YES (fluctuations on all scales)
\item Near bifurcation? YES (second-order phase transition)
\item Universal exponents needed? YES (experimental predictions)
\item $\Rightarrow$ Full RG with Wilson-Fisher fixed point analysis required.
\end{itemize}

\textbf{Example 4: Porous medium equation with $m = 1.1$}
\begin{itemize}
\item Scale hierarchy? YES (initial width vs late-time spread)
\item Pathologies? YES (dimensional analysis fails)
\item Near bifurcation? NO (smooth transition at $m = 1$)
\item Universal exponents? YES (anomalous Barenblatt exponent)
\item $\Rightarrow$ RG for anomalous dimensions; exact solution exists here.
\end{itemize}
\end{workedbox}

%-------------------------------------------------------------------------------
\section{The Sethna Problem-Solving Template}
\label{sec:sethna_template}
%-------------------------------------------------------------------------------

For problems where RG \emph{is} needed, Sethna advocates a systematic approach. Before diving into calculations, answer three fundamental questions.

\marginnote{Sethna's template: identify order parameter, symmetry, and topology before computing.}

\subsection{What Is the Order Parameter?}

The order parameter determines the \emph{coordinates on theory space} $\MM$:
\begin{center}
\renewcommand{\arraystretch}{1.2}
\begin{tabular}{lll}
\textbf{System} & \textbf{Order Parameter} & \textbf{Theory Space Coords} \\
\hline
Ferromagnet & Magnetization $M$ & $(T - T_c, h, \ldots)$ \\
Superfluid & $\psi = |\psi|e^{i\theta}$ & $(T - T_\lambda, \mu, \ldots)$ \\
Ising model & Spin density $\sigma$ & $(K - K_c, H, \ldots)$ \\
Fluid turbulence & Velocity field $\mathbf{u}$ & (Re, geometry) \\
\end{tabular}
\end{center}

\subsection{What Symmetry Is Broken?}

The broken symmetry determines the \emph{group structure} of the RG:
\begin{center}
\renewcommand{\arraystretch}{1.2}
\begin{tabular}{lll}
\textbf{Transition} & \textbf{Broken Symmetry} & \textbf{Universality Class} \\
\hline
Ferromagnetic (uniaxial) & $\mathbb{Z}_2$ & Ising \\
Ferromagnetic (isotropic) & $O(3)$ & Heisenberg \\
Superfluid/superconductor & $U(1)$ & XY \\
Crystallization & Translation & Solid \\
\end{tabular}
\end{center}

\subsection{What Are the Topological Defects?}

Topological defects correspond to \emph{singular points or surfaces} in theory space:
\begin{center}
\renewcommand{\arraystretch}{1.2}
\begin{tabular}{llll}
\textbf{System} & \textbf{Order Space} & $\boldsymbol{\pi_1}$ & \textbf{Defects} \\
\hline
2D XY model & $S^1$ & $\mathbb{Z}$ & Vortices \\
3D Heisenberg & $S^2$ & 0 & None (monopoles from $\pi_2$) \\
Nematic & $\mathbb{RP}^2$ & $\mathbb{Z}_2$ & Half-integer disclinations \\
Crystal & $T^3$ & $\mathbb{Z}^3$ & Dislocations \\
\end{tabular}
\end{center}

\textbf{Only after answering these questions should you begin detailed calculations.}

This discipline prevents common errors: computing without understanding what the order parameter is, missing symmetry-protected features, or overlooking topological contributions to the partition function.

%-------------------------------------------------------------------------------
\section{UV Divergences and Regularization}
\label{sec:regularization}
%-------------------------------------------------------------------------------

\marginnote{Regularization makes divergent integrals finite. Renormalization then absorbs the divergences into redefined parameters. These are distinct operations.}

Before perturbation theory can produce even a divergent series, we must first deal with a more immediate problem: individual Feynman diagrams often involve \emph{divergent integrals}. These ultraviolet (UV) divergences arise from loop momenta that extend to infinity. \textbf{Regularization} is the process of introducing a parameter that renders these integrals finite, allowing us to manipulate them algebraically before ultimately removing the regulator.

\subsection{The Need for Regularization}

Consider the simplest divergent integral in four-dimensional quantum field theory: the one-loop correction to the scalar propagator in $\phi^4$ theory. The self-energy diagram gives:
\begin{equation}
\Sigma(p^2) = \frac{\lambda}{2} \int \frac{d^4k}{(2\pi)^4} \frac{1}{k^2 + m^2}
\end{equation}
This integral diverges quadratically: as $k \to \infty$, the integrand behaves as $1/k^2$, giving $\int^\Lambda k\,dk \sim \Lambda^2$.

\textbf{The solution:} Introduce a \emph{regulator} that makes the integral finite, compute the result as a function of that parameter, and then carefully take the limit where the regulator is removed. The divergences that appear are absorbed into redefinitions of physical parameters---this is renormalization.

\subsection{Dimensional Regularization}
\label{sec:dim_reg}

The most powerful regularization method is \textbf{dimensional regularization}, which analytically continues the number of spacetime dimensions from 4 to $d = 4 - \epsilon$.

\marginnote{Dimensional regularization was developed by 't~Hooft and Veltman (1972) for gauge theories.}

The key features are:
\begin{itemize}
\item \textbf{Preserves gauge invariance:} No explicit cutoff breaks symmetry.
\item \textbf{Algebraically simple:} Divergences appear as $1/\epsilon$ poles.
\item \textbf{No power-law divergences:} Scaleless integrals vanish by definition.
\end{itemize}

\subsection{Other Regularization Methods}

\textbf{Pauli-Villars:} Modifies propagators by introducing fictitious heavy particles. Preserves Lorentz and gauge invariance in QED.

\textbf{Zeta function:} Uses analytic continuation of sums. Elegant for Casimir-type calculations.

\textbf{Lattice:} Discretizes spacetime. Essential for non-perturbative calculations.

The key principle is that \emph{physical predictions are regularization-independent}. Different schemes give different intermediate expressions, but after renormalization, all observables agree.

%-------------------------------------------------------------------------------
\section{Renormalization Schemes}
\label{sec:renorm_schemes}
%-------------------------------------------------------------------------------

Once divergences are regulated, they must be absorbed into redefinitions of parameters. The precise way finite parts are treated defines a \textbf{renormalization scheme}.

\marginnote{Renormalization absorbs divergences into redefined parameters. The scheme specifies how finite parts are handled.}

\subsection{The On-Shell Scheme}

The \textbf{on-shell scheme} defines renormalized parameters to equal directly measurable physical quantities. For QED, the renormalized mass and charge are exactly the physical electron mass and charge.

\textbf{Advantages:} Parameters have direct physical meaning.

\textbf{Disadvantages:} IR divergences for massless theories; complexity at higher orders.

\subsection{Minimal Subtraction: MS and $\overline{\text{MS}}$}

\textbf{Minimal subtraction (MS)} works with dimensional regularization, subtracting only the $1/\epsilon$ poles. The \textbf{$\overline{\text{MS}}$} scheme also subtracts $\gamma_E - \ln 4\pi$.

\marginnote{$\overline{\text{MS}}$ is the standard scheme for QCD calculations.}

\textbf{Advantages:} Computational simplicity; preserves symmetries; mass-independent.

\textbf{Disadvantages:} Parameters not directly physical.

\subsection{Scheme Independence}

A fundamental result is that \emph{physical observables are scheme-independent}. Different schemes are different coordinate systems on theory space $\mathcal{M}$. Physical quantities are geometric invariants.

The first two coefficients of the beta function, $\beta_0$ and $\beta_1$, are universal across mass-independent schemes.

%-------------------------------------------------------------------------------
\section{Summary and Road Ahead}
\label{sec:road_to_part2}
%-------------------------------------------------------------------------------

This chapter has covered the foundations of perturbation theory:

\begin{enumerate}
\item Perturbation series generically diverge with factorial ($n!$) growth---Gevrey-1 structure.
\item The divergence encodes non-perturbative physics (developed in Chapter~\ref{ch:resurgence}).
\item The same mathematical structure appears in ODEs, PDEs, and QFT.
\item UV divergences in loop integrals require regularization and renormalization.
\item Physical predictions are independent of regularization and renormalization scheme.
\end{enumerate}

The next two chapters complete the machinery:
\begin{itemize}
\item \textbf{Chapter~\ref{ch:resurgence}}: Resurgence---extracting non-perturbative physics from divergent series
\item \textbf{Chapter~\ref{ch:algebra}}: The deeper algebraic structure---Hopf algebras and Riemann-Hilbert
\end{itemize}

%-------------------------------------------------------------------------------
\section*{Exercises}
\addcontentsline{toc}{section}{Exercises}
%-------------------------------------------------------------------------------

\begin{enumerate}
\item \textbf{Identifying scales.} For each of the following systems, identify the scales and describe the scale hierarchy:
\begin{enumerate}
\item A pendulum with small amplitude oscillations and weak damping.
\item Heat conduction in a rod with both ends at different fixed temperatures.
\item The quantum double-well potential $V(x) = \lambda(x^2 - a^2)^2$.
\end{enumerate}

\item \textbf{Applying the decision tree.} For each system below, work through the decision tree to determine whether RG methods are needed:
\begin{enumerate}
\item A damped driven pendulum far from resonance.
\item The Navier-Stokes equations at Reynolds number $\text{Re} = 10$.
\item The Navier-Stokes equations at $\text{Re} = 10^6$.
\item A polymer chain in good solvent.
\end{enumerate}

\item \textbf{Factorial growth.} The solution to $\epsilon y' + y = 1$ with $y(0) = 0$ has the exact form $y(x) = 1 - e^{-x/\epsilon}$.
\begin{enumerate}
\item Expand $y(x)$ in powers of $\epsilon$ to find the formal series.
\item Show that the coefficients grow factorially.
\item Verify that the series is Gevrey-1.
\item Explain why truncating the series at any finite order fails to capture the exponentially small term $e^{-x/\epsilon}$.
\end{enumerate}

\item \textbf{Boundary layer.} Consider the boundary layer equation $\epsilon y'' + y' + y = 0$ with $y(0) = 0$, $y(1) = 1$.
\begin{enumerate}
\item Identify the outer and inner solutions.
\item Show that the outer solution has secular behavior near $x = 0$.
\item Set up the matched asymptotic expansion and identify the running parameter.
\end{enumerate}

\item \textbf{Sethna template.} Apply the Sethna problem-solving template to the following systems:
\begin{enumerate}
\item Liquid-gas critical point.
\item Antiferromagnetic Ising model.
\item Cholesteric liquid crystal.
\end{enumerate}

\item \textbf{(Challenge) Van der Pol divergence.} For the Van der Pol oscillator $\ddot{x} + \epsilon(x^2-1)\dot{x} + x = 0$:
\begin{enumerate}
\item Set up the multiple-scales expansion for the limit cycle amplitude.
\item Compute the first three terms in the series $A = 2 + a_1\epsilon + a_2\epsilon^2 + \cdots$.
\item Argue on physical grounds why the series must diverge for large $\epsilon$ (hint: relaxation oscillations).
\end{enumerate}

\item \textbf{(Challenge) Instanton action.} For the double-well potential $V(x) = \frac{\lambda}{4}(x^2 - a^2)^2$:
\begin{enumerate}
\item Find the classical instanton solution interpolating between the two minima.
\item Compute the instanton action $S_{\text{inst}} = \int_{-\infty}^{\infty} \frac{1}{2}\dot{x}^2 + V(x)\,dt$.
\item Explain why this action appears in the large-order behavior of the ground state energy expansion.
\end{enumerate}
\end{enumerate}

