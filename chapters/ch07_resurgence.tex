%===============================================================================
\chapter{The Unified Recipe}
\label{ch:recipe}
%===============================================================================

\marginnote{This chapter distills the unified framework into a practical methodology. There is no separate ``resurgence step'' because resurgent thinking has been present from the beginning. The recipe shows how perturbative and non-perturbative analysis are inseparable aspects of one approach.}

The preceding chapters developed the unified geometric-resurgent framework for the renormalization group. Parameter space is a manifold with metric and connection structure. Perturbation series are asymptotic with Gevrey-1 divergence encoding non-perturbative physics through the Borel plane. Stokes phenomena are monodromy in the extended theory space. Fixed points can be perturbative or non-perturbative.

This final chapter of Part I synthesizes everything into a practical \textbf{methodology} that can be applied to new problems. The recipe has six steps, but unlike traditional treatments, there is no separate ``resurgence step'' tacked on at the end. The transseries structure and Borel analysis appear naturally throughout.

%-------------------------------------------------------------------------------
\section{The Six Steps}
\label{sec:six_steps}
%-------------------------------------------------------------------------------

The unified RG analysis proceeds as follows.

\subsection{Step 1: Identify Scales and Recognize Divergence Structure}

Every RG problem begins with a hierarchy of scales. Identify the UV and IR scales, the small parameter $\epsilon$ connecting them, and the regime where naive analysis fails.

\marginnote{Step 1: Find the scales, the small parameter, and where naive perturbation theory breaks down. Recognize that the perturbative series will diverge with Gevrey-1 structure.}

Simultaneously, recognize that the perturbative expansion in $\epsilon$ will diverge. Perturbation series in physical problems generically have factorially growing coefficients, which is Gevrey-1 structure. This is not a failure but a feature because the pattern of divergence encodes non-perturbative physics.

\textbf{Questions to ask:}
What are the separated scales?
What small parameter relates them?
Where does naive perturbation theory fail (secular terms, UV divergences, boundary mismatches)?
What is the expected source of non-perturbative effects (instantons, tunneling, renormalons)?

\textbf{For the oscillator:} The scales are the oscillation period $1/\omega_0$ and the amplitude-drift time $\omega_0/(\lambda A^2)$. The small parameter is $\lambda A^2/\omega_0^2$. Naive perturbation theory fails at $t \sim 1/(\lambda A^2)$ with secular terms. Non-perturbative effects come from complex-time instantons.

\textbf{For 1D $\phi^4$:} The scales are the UV cutoff $\Lambda$ and the mass scale $\sqrt{r}$. The small parameter is $\lambda/\Lambda^2$. Loop integrals are finite in 1D, but the perturbative series still diverges. Non-perturbative effects are renormalons from RG running.

\textbf{For the PME:} The scales are the initial localization width and the time-evolved width. The small parameter is $(m-1)$ measuring deviation from linear diffusion. The perturbative expansion in $(m-1)$ is asymptotic. The selection of the physical exponent involves resummation.

\subsection{Step 2: Set Up Perturbation Theory and Borel Transform}

Construct the formal perturbative expansion. Compute the first few orders and identify the structure of divergence (alternating signs, specific factorial growth rate). Immediately construct the Borel transform to see the singularity structure.

\marginnote{Step 2: Compute the perturbative series, take its Borel transform, and identify the singularities. The singularities correspond to non-perturbative sectors.}

\textbf{Questions to ask:}
What is the perturbative series?
What is the pattern of coefficients?
Where are the singularities in the Borel plane?
What physical effects do these singularities correspond to (instantons, renormalons, etc.)?

\textbf{For the oscillator:} The perturbative frequency correction is $\omega = \omega_0(1 + \frac{3}{8}\Pi + c_2\Pi^2 + \cdots)$ where $\Pi = \lambda A^2/\omega_0^2$. The Borel transform has singularities related to the complex-time instanton action $S = \omega_0^3/(3\lambda)$.

\textbf{For 1D $\phi^4$:} The perturbative beta functions are asymptotic series. The Borel transform of $\beta_\lambda$ has renormalon singularities at $\zeta_k = k/\beta_1 = k/2$ for positive integer $k$.

\textbf{For the PME:} The exponent $\beta(m)$ expanded around $m = 1$ is an asymptotic series. The Borel transform has singularities corresponding to sub-leading self-similar modes that were discarded in selecting the Barenblatt solution.

\subsection{Step 3: Identify Running Parameters Including Transseries}

Determine which quantities must become scale-dependent to absorb the divergence. These are the ``running parameters.'' Include not only the perturbative couplings but also the transseries parameters that weight non-perturbative sectors.

\marginnote{Step 3: Identify which parameters run, including the transseries weights. The full parameter space is the extended space $(g^a, \sigma^n)$.}

\textbf{Questions to ask:}
Which parameters absorb secular terms or divergences?
What are the transseries parameters $\sigma^n$?
What is the full extended parameter space?

\textbf{For the oscillator:} The running parameters are $A(t)$ and $\phi(t)$. The extended space adds $\sigma$ weighting the instanton sector: $(A, \phi, \sigma)$.

\textbf{For 1D $\phi^4$:} The running parameters are $r(\mu)$ and $\lambda(\mu)$. The extended space adds renormalon transseries parameters: $(r, \lambda, \sigma_{\text{ren}})$.

\textbf{For the PME:} The running parameter is the effective scaling exponent. The extended space includes parameters selecting among possible self-similar modes.

\subsection{Step 4: Derive Full Transseries Beta Functions}

Derive the RG equations for all parameters, including transseries coordinates. Use the envelope method or Callan-Symanzik approach. The consistency requirement across Stokes lines automatically produces the bridge equations.

\marginnote{Step 4: Derive RG equations for all coordinates in extended space. Consistency at Stokes crossings produces the bridge equations of alien calculus.}

\textbf{Questions to ask:}
What are the beta functions for perturbative couplings?
What are the beta functions for transseries parameters?
Where are the Stokes lines?
What are the Stokes constants (monodromy data)?

\textbf{For the oscillator:}
\begin{align}
\beta_A &= 0 + O(\sigma e^{-S/\lambda}) \\
\beta_\phi &= \frac{3\lambda A^2}{8\omega_0} + O(\sigma e^{-S/\lambda})
\end{align}
The transseries corrections are exponentially small. The Stokes constant $S_1$ is determined by the instanton calculation.

\textbf{For 1D $\phi^4$:}
\begin{align}
\beta_r &= 2r + \frac{3\lambda\Lambda}{\pi(\Lambda^2 + r)} + O(\sigma_{\text{ren}}e^{-\zeta_1/\lambda}) \\
\beta_\lambda &= 2\lambda + O(\sigma_{\text{ren}}e^{-\zeta_1/\lambda})
\end{align}
The renormalon Stokes constant is related to the one-loop beta function.

\textbf{For the PME:} The beta function for the exponent is trivial (exponents don't run once determined), but the selection involves the resummation prescription.

\subsection{Step 5: Analyze Flow Structure in Extended Space}

Find all fixed points, including perturbative and non-perturbative ones. Classify stability using the eigenvalues of the full stability matrix on extended space. Identify universality classes.

\marginnote{Step 5: Find fixed points, classify stability, and identify universality. Include non-perturbative fixed points where $\beta_{\text{pert}} \neq 0$ but $\beta_{\text{full}} = 0$.}

\textbf{Questions to ask:}
Where are the fixed points of $\beta_{\text{pert}}$?
Are there non-perturbative fixed points where $\beta_{\text{pert}} \neq 0$ but $\beta_{\text{full}} = 0$?
What is the stability matrix?
Which directions are relevant, irrelevant, marginal?
What is the basin of attraction?

\textbf{For the oscillator:} The only fixed point is $A = 0$ (trivial). All trajectories with $A > 0$ flow forever without reaching a non-trivial fixed point.

\textbf{For 1D $\phi^4$:} The Gaussian fixed point $(0, 0)$ is the only perturbative fixed point. Both directions are relevant (unstable). No non-perturbative fixed points are known in 1D.

\textbf{For the PME:} The Barenblatt profile is a stable attractor in the space of self-similar solutions. The exponents are uniquely determined for $m > 1$.

\subsection{Step 6: Extract Physics Using Median Resummation}

Physical predictions come from correctly resummed transseries, not from naive perturbative truncation. Use median resummation or related prescriptions to obtain real, unambiguous answers.

\marginnote{Step 6: Extract physical predictions by resumming the full transseries. Median resummation gives real, unambiguous results.}

\textbf{Questions to ask:}
What physical observable are we computing?
How does it depend on the resummation prescription?
What is the median resummation?
Are there ambiguities, and how do they cancel?

\textbf{For the oscillator:} The physical frequency is
\begin{equation}
\omega_{\text{eff}} = \omega_0 + \frac{3\lambda A^2}{8\omega_0} + O(\lambda^2)
\end{equation}
Higher-order corrections require resummation for quantitative accuracy at large $\lambda$.

\textbf{For 1D $\phi^4$:} Physical quantities like the correlation length are computed from the resummed running couplings. Renormalon ambiguities cancel in physical observables.

\textbf{For the PME:} The physical Barenblatt exponent
\begin{equation}
\beta = \frac{1}{d(m-1) + 2}
\end{equation}
is exact (the PME is special). For more general problems, the exponent would require resummation.

%-------------------------------------------------------------------------------
\section{The Recipe Applied: Anharmonic Oscillator}
\label{sec:recipe_oscillator}
%-------------------------------------------------------------------------------

Let's walk through the complete recipe for the anharmonic oscillator.

\begin{workedbox}[Box 7.1: Complete Analysis of the Anharmonic Oscillator]
\textbf{Step 1: Scales and divergence.}
UV scale: oscillation period $\tau_{\text{fast}} \sim 1/\omega_0$.
IR scale: amplitude-drift time $\tau_{\text{slow}} \sim \omega_0/(\lambda A^2)$.
Small parameter: $\epsilon = \lambda A^2/\omega_0^2 \ll 1$.
Breakdown: secular terms at $t \sim \tau_{\text{slow}}$.
Non-perturbative: complex-time instantons.

\textbf{Step 2: Perturbation theory and Borel.}
The perturbative solution $x(t) = A\cos(\omega_0 t) + O(\lambda)$ develops secular terms.
The frequency series $\omega = \omega_0(1 + \frac{3}{8}\epsilon + c_2\epsilon^2 + \cdots)$ diverges with $|c_n| \sim n!$.
The Borel transform $\hat{\omega}(\zeta)$ has singularities at $\zeta = \omega_0^3/(3\lambda)$ (instanton action).

\textbf{Step 3: Running parameters.}
Perturbative: $(A, \phi)$.
Extended: $(A, \phi, \sigma)$ with $\sigma$ weighting instanton sector.

\textbf{Step 4: Beta functions.}
\begin{align}
\frac{dA}{dt} &= 0 \\
\frac{d\phi}{dt} &= \frac{3\lambda A^2}{8\omega_0}
\end{align}
Transseries corrections: $O(\sigma e^{-S/\lambda})$.
The Stokes constant $S_1$ relates perturbative and instanton sectors.

\textbf{Step 5: Fixed points and stability.}
Perturbative fixed point: $A = 0$ (trivial).
No non-perturbative fixed points.
All $A > 0$ trajectories flow forever.

\textbf{Step 6: Physical prediction.}
The effective frequency is:
\begin{equation}
\omega_{\text{eff}} = \omega_0\left(1 + \frac{3\lambda A^2}{8\omega_0^2}\right) + O(\lambda^2)
\end{equation}
For quantitative accuracy at larger $\lambda$, resum using median prescription.
\end{workedbox}

%-------------------------------------------------------------------------------
\section{The Recipe Applied: 1D $\phi^4$ Theory}
\label{sec:recipe_phi4}
%-------------------------------------------------------------------------------

The field theory example demonstrates statistical RG with non-trivial flow.

\begin{workedbox}[Box 7.2: Complete Analysis of 1D $\phi^4$ Theory]
\textbf{Step 1: Scales and divergence.}
UV scale: cutoff $\Lambda$ (lattice spacing).
IR scale: correlation length $\xi \sim 1/\sqrt{r}$.
Small parameter: $\lambda/\Lambda^2 \ll 1$.
Breakdown: tadpole corrections grow with $\Lambda$.
Non-perturbative: renormalons from RG running.

\textbf{Step 2: Perturbation theory and Borel.}
The beta functions $\beta_r = 2r + 3\lambda\Lambda/\pi(\Lambda^2 + r)$ and $\beta_\lambda = 2\lambda$ are perturbative leading terms.
Higher-order coefficients grow factorially.
The Borel transform has renormalon singularities at $\zeta_k = k/2$.

\textbf{Step 3: Running parameters.}
Perturbative: $(r, \lambda)$.
Extended: $(r, \lambda, \sigma_{\text{ren}})$.

\textbf{Step 4: Beta functions.}
\begin{align}
\beta_r &= 2r + \frac{3\lambda\Lambda}{\pi(\Lambda^2 + r)} + O(\sigma_{\text{ren}}e^{-1/(2\lambda)}) \\
\beta_\lambda &= 2\lambda + O(\sigma_{\text{ren}}e^{-1/(2\lambda)})
\end{align}
The renormalon Stokes constant: $S_{\text{ren}} = 1/\beta_1 + O(1) = 1/2 + O(1)$.

\textbf{Step 5: Fixed points and stability.}
Perturbative: Gaussian fixed point $(0, 0)$.
Stability matrix eigenvalues: both $= 2$ (relevant, unstable).
No non-perturbative fixed points in 1D.
In $d = 4 - \epsilon$, the Wilson-Fisher fixed point appears.

\textbf{Step 6: Physical prediction.}
Running couplings:
\begin{equation}
\lambda(\mu) = \lambda_0\left(\frac{\mu}{\mu_0}\right)^2
\end{equation}
Physical correlation functions computed from resummed expressions.
\end{workedbox}

%-------------------------------------------------------------------------------
\section{The Recipe Applied: Porous Medium Equation}
\label{sec:recipe_pme}
%-------------------------------------------------------------------------------

The PME demonstrates anomalous dimensions and second-kind self-similarity.

\begin{workedbox}[Box 7.3: Complete Analysis of the Porous Medium Equation]
\textbf{Step 1: Scales and divergence.}
UV scale: initial localization width.
IR scale: late-time spread $L(t) \sim t^\beta$.
Small parameter: $(m - 1)$ (deviation from linear diffusion).
Breakdown: anomalous exponent $\beta \neq 1/2$ for $m \neq 1$.
Non-perturbative: sub-leading self-similar modes.

\textbf{Step 2: Perturbation theory and Borel.}
Expand $\beta(m)$ around $m = 1$:
\begin{equation}
\beta = \frac{1}{2} - \frac{d}{4}(m-1) + O((m-1)^2)
\end{equation}
This is asymptotic with singularities corresponding to competing modes.

\textbf{Step 3: Running parameters.}
The exponent $\beta$ is determined by the self-similar ansatz.
Extended space includes mode weights selecting among solutions.

\textbf{Step 4: Selection principle.}
Mass conservation: $\alpha = d\beta$.
Self-consistency: $\beta(md + 2 - d) = 1$.
Result:
\begin{equation}
\beta = \frac{1}{d(m-1) + 2}
\end{equation}
The physical mode is selected by boundary conditions (finite mass, compact support).

\textbf{Step 5: Fixed points and stability.}
The Barenblatt profile is the unique stable self-similar attractor.
Other self-similar modes exist but are unstable.

\textbf{Step 6: Physical prediction.}
The late-time density profile:
\begin{equation}
\rho(x, t) = \frac{1}{t^\alpha}\left[C - \frac{(m-1)}{4md}\frac{|x|^2}{(Dt)^{2\beta}}\right]_+^{1/(m-1)}
\end{equation}
This is exact for the PME. More general nonlinear diffusion would require resummation.
\end{workedbox}

%-------------------------------------------------------------------------------
\section{When to Trust Perturbation Theory}
\label{sec:when_trust}
%-------------------------------------------------------------------------------

The unified framework includes both perturbative and non-perturbative physics, but in many practical situations perturbation theory alone is sufficient.

\subsection{Conditions for Perturbative Accuracy}

Perturbation theory gives accurate answers when the coupling is small ($\epsilon \ll 1$), no Stokes lines are crossed in the physical region, and we stay near a perturbative fixed point.

\marginnote{Perturbation theory works when you're far from Stokes lines and close to a perturbative fixed point with small coupling.}

Under these conditions, the exponentially suppressed transseries corrections $e^{-S/\epsilon}$ are genuinely negligible. Computing to order $\epsilon^n$ gives accuracy $O(\epsilon^{n+1})$ as expected.

\subsection{When Full Analysis Is Required}

The full resurgent analysis becomes necessary when the coupling is not small, Stokes lines are crossed (e.g., analytic continuation in parameters), we approach non-perturbative fixed points, or ambiguities must cancel for physical predictions.

In these situations, truncating the perturbative series can give qualitatively wrong answers. The transseries structure is essential.

\subsection{The Oscillator at Strong Coupling}

For the anharmonic oscillator at large $\lambda A^2/\omega_0^2$, perturbation theory breaks down. The frequency correction is no longer well-approximated by the leading term. Resummation of the full series, including proper treatment of Stokes phenomena, is required for accurate predictions.

\subsection{Critical Phenomena}

Near critical points (phase transitions), fluctuations are large and perturbation theory in the original couplings fails. However, the epsilon expansion around $d = 4$ can be perturbative in $\epsilon$. Even then, accurate exponents at $\epsilon = 1$ (3D) require resummation.

%-------------------------------------------------------------------------------
\section{Common Pitfalls}
\label{sec:pitfalls}
%-------------------------------------------------------------------------------

Several common errors can derail an RG analysis.

\subsection{Ignoring Divergence Structure}

Treating perturbative series as convergent and simply truncating at some order ignores the information encoded in the divergence pattern. For asymptotic series, optimal truncation (stopping at the smallest term) gives exponentially small error, but not the full answer.

\marginnote{Ignoring divergence structure throws away non-perturbative information encoded in the pattern of coefficients.}

\subsection{Missing Stokes Lines}

When continuing analytically in parameters (complex coupling, etc.), Stokes lines may be crossed. Ignoring the resulting jumps in transseries parameters leads to wrong answers in the new region.

\subsection{Confusing Scheme Dependence with Physics}

Beta functions and anomalous dimensions are scheme-dependent. Only scheme-independent quantities (critical exponents, Stokes constants, physical observables) are meaningful. Comparing beta functions in different schemes without accounting for the scheme transformation is a common error.

\subsection{Overlooking Non-Perturbative Fixed Points}

If only perturbative fixed points are sought, non-perturbative ones are missed. For some problems, the physically relevant fixed point may be non-perturbative.

%-------------------------------------------------------------------------------
\section{The Three Examples in Perspective}
\label{sec:examples_perspective}
%-------------------------------------------------------------------------------

The three canonical examples were chosen to illustrate different aspects of the unified framework.

\textbf{The anharmonic oscillator} is the simplest example and suffices to demonstrate secular terms and running parameters, the resolution via RG equations, Gevrey-1 divergence and the Borel plane, and the basic transseries structure.

It is too simple for non-trivial fixed points, operator mixing or anomalous dimensions, and statistical RG with coarse-graining.

\marginnote{The three examples form a ladder: oscillator $\to$ field theory $\to$ PDE. Each adds capabilities the previous lacked.}

\textbf{The 1D $\phi^4$ theory} adds non-trivial beta functions with multiple couplings, the Gaussian fixed point and its stability, renormalon singularities from RG running, and statistical mechanics interpretation.

It is still too simple for non-trivial interacting fixed points (which require $d < 4$) and anomalous dimensions.

\textbf{The porous medium equation} adds anomalous dimensions (second-kind self-similarity), non-trivial scaling exponents from dynamics, Wasserstein gradient flow structure, and selection principles for physical solutions.

Together, the three examples demonstrate the complete framework. Any new problem will share features with one or more of these examples, and the techniques transfer accordingly.

%-------------------------------------------------------------------------------
\section{Transition to Part II}
\label{sec:transition}
%-------------------------------------------------------------------------------

Part I has developed the theoretical framework. Part II applies it to specific physical systems.

Each application chapter will use the six-step recipe and identify scales, set up perturbation theory with Borel analysis, find running parameters including transseries, derive beta functions and flow equations, analyze fixed points and universality, and extract physical predictions with proper resummation.

\marginnote{Part II applications: each physical system analyzed using the unified recipe developed in Part I.}

The applications span diverse areas including chaotic dynamics (Lorenz system), fluid mechanics (Navier-Stokes), solid mechanics (fracture), statistical mechanics (Ising, O(N) models), quantum field theory (QED), and condensed matter (Hubbard model).

In each case, the unified geometric-resurgent perspective reveals structure that would be invisible to purely perturbative analysis.

%-------------------------------------------------------------------------------
\section*{Summary of Part I}
\addcontentsline{toc}{section}{Summary of Part I}
%-------------------------------------------------------------------------------

\begin{center}
\fbox{\parbox{0.85\textwidth}{
\textbf{The renormalization group} is a framework for analyzing systems with scale hierarchy. Parameters that appear constant become scale-dependent when we properly handle secular terms, divergences, or boundary mismatches.

\textbf{Perturbation series diverge with Gevrey-1 structure.} The Borel transform converts factorial growth to geometric, revealing singularities that encode non-perturbative physics (instantons, renormalons).

\textbf{The transseries} $\tilde{f}(g, \sigma) = f_0(g) + \sum_n \sigma^n e^{-nS/g}f_n(g)$ is the complete solution including all non-perturbative sectors.

\textbf{Parameter space is a manifold} with coordinates $(g^a, \sigma^n)$ combining perturbative couplings and transseries parameters. It has metric and connection structure.

\textbf{The beta function} $\boldsymbol{\beta}$ is the generator of scale transformations. Fixed points satisfy $\beta = 0$ and can be perturbative or non-perturbative.

\textbf{Stokes phenomena are monodromy} in the extended parameter space. The alien derivative is a covariant derivative probing non-perturbative directions.

\textbf{The unified recipe:}
\begin{enumerate}
\item Identify scales and divergence structure
\item Set up perturbation theory and Borel transform
\item Identify running parameters including transseries
\item Derive full beta functions with Stokes consistency
\item Analyze flow in extended space
\item Extract physics via median resummation
\end{enumerate}

\textbf{The three examples} demonstrate the framework:
The oscillator shows secular terms and running parameters.
The $\phi^4$ theory shows beta functions and fixed points.
The PME shows anomalous dimensions.
}}
\end{center}
