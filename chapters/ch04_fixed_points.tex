%===============================================================================
\chapter{Fixed Points, Stability, and the Transseries Landscape}
\label{ch:fixed_points}
%===============================================================================

\marginnote{Chapters 1--3 showed that parameters run with scale. This chapter asks: where do they run \emph{to}? Fixed points are the destinations, and their classification determines the qualitative behavior of physical systems. We treat perturbative and non-perturbative fixed points on equal footing.}

The RG generates flows on parameter space. But flows go somewhere. The \textbf{destinations} of RG flows are called \textbf{fixed points}, and they represent theories that are exactly scale-invariant. Understanding fixed points is the key to understanding the long-distance or long-time behavior of any system.

This chapter develops the theory of fixed points in depth. We classify operators near fixed points as \textbf{relevant}, \textbf{irrelevant}, or \textbf{marginal} based on how perturbations grow or shrink under RG. We introduce the concept of \textbf{universality classes}, which is the remarkable fact that different microscopic theories can flow to the same fixed point. And we introduce our third example, the \textbf{porous medium equation}, which exhibits \textbf{anomalous dimensions} that cannot be predicted by dimensional analysis.

Crucially, we treat fixed points of the full transseries beta function, not just the perturbative piece. This leads to the possibility of \textbf{non-perturbative fixed points} where the perturbative beta function is nonzero but non-perturbative corrections cancel it. Such fixed points are completely invisible to perturbation theory yet can control the physics of real systems.

%-------------------------------------------------------------------------------
\section{Perturbative Fixed Points}
\label{sec:perturbative_fp}
%-------------------------------------------------------------------------------

A fixed point is where the RG flow stops. At a fixed point, the theory doesn't change under scale transformations.

\subsection{Definition}

\marginnote{At a fixed point, all beta functions vanish. The theory is exactly scale-invariant.}

A \textbf{perturbative fixed point} is a point $g^* = (g^{*1}, \ldots, g^{*n})$ where all perturbative beta functions vanish:
\begin{equation}
\beta^i_{\text{pert}}(g^*) = 0 \quad \text{for all } i
\end{equation}

At such a point, the running stops because $dg^i/d\ell = 0$. The couplings take the same values at all scales.

\subsection{Examples We've Seen}

\textbf{The anharmonic oscillator} has $\beta^A = 0$ and $\beta^\phi = 3\lambda A^2/(8\omega_0)$. The fixed point $A^* = 0$ corresponds to the oscillator at rest.

\textbf{The 1D $\phi^4$ theory} has the Gaussian fixed point $(r^*, \lambda^*) = (0, 0)$, which is free field theory.

Both are \textbf{trivial} fixed points in the sense that the interactions have vanished. More interesting are fixed points with $\lambda^* \neq 0$.

\subsection{The Wilson-Fisher Fixed Point}

In $d = 4 - \epsilon$ dimensions, $\phi^4$ theory has a famous non-trivial fixed point discovered by Wilson and Fisher. The beta function for the quartic coupling takes the form:
\begin{equation}
\beta_\lambda = -\epsilon\lambda + b\lambda^2 + O(\lambda^3)
\end{equation}
where the coefficient $b > 0$.

\marginnote{The Wilson-Fisher fixed point controls phase transitions in real 3D systems. It is perturbatively accessible in $d = 4 - \epsilon$.}

Setting $\beta_\lambda = 0$ gives fixed points at $\lambda^* = 0$ (Gaussian) and:
\begin{equation}
\lambda^*_{\text{WF}} = \frac{\epsilon}{b} + O(\epsilon^2)
\end{equation}

This Wilson-Fisher fixed point is non-trivial because $\lambda^*_{\text{WF}} \neq 0$. It describes the universality class of the Ising model in $d = 3$ (setting $\epsilon = 1$).

\begin{workedbox}[Box 4.1: Stability of the Wilson-Fisher Fixed Point]
\textbf{Setup:} The beta function in $d = 4 - \epsilon$ is $\beta_\lambda = -\epsilon\lambda + b\lambda^2 + O(\lambda^3)$.

\textbf{The fixed points:}
Gaussian fixed point at $\lambda^*_G = 0$. Wilson-Fisher fixed point at $\lambda^*_{\text{WF}} = \epsilon/b + O(\epsilon^2)$.

\textbf{Stability analysis:} Linearize $\beta_\lambda$ around each fixed point.

\underline{At the Gaussian:}
\begin{equation}
\frac{d(\delta\lambda)}{d\ell} = \left.\frac{d\beta_\lambda}{d\lambda}\right|_{\lambda=0}\delta\lambda = -\epsilon\,\delta\lambda
\end{equation}
The eigenvalue is $-\epsilon < 0$ (for $\epsilon > 0$), so perturbations shrink. The Gaussian is \textbf{stable} (IR attractive).

\underline{At Wilson-Fisher:}
\begin{equation}
\frac{d(\delta\lambda)}{d\ell} = \left.\frac{d\beta_\lambda}{d\lambda}\right|_{\lambda^*}\delta\lambda = (-\epsilon + 2b\lambda^*)\delta\lambda = \epsilon\,\delta\lambda
\end{equation}
The eigenvalue is $+\epsilon > 0$, so perturbations grow. Wilson-Fisher is \textbf{unstable} (UV attractive).

\textbf{Physical picture:} The flow goes from Wilson-Fisher (UV) to Gaussian (IR). Theories near Wilson-Fisher flow toward free theory at long distances. The WF fixed point controls the approach to criticality.
\end{workedbox}

%-------------------------------------------------------------------------------
\section{Non-Perturbative Fixed Points}
\label{sec:nonpert_fp}
%-------------------------------------------------------------------------------

The perturbative fixed points are not the full story. When we consider the complete transseries beta function, new fixed points can emerge that are invisible to perturbation theory.

\subsection{The Full Fixed Point Condition}

A \textbf{fixed point of the full theory} satisfies:
\begin{equation}
\beta^i_{\text{full}}(g^*, \sigma^*) = 0 \quad \text{for all } i
\end{equation}
where $\beta_{\text{full}}$ is the complete transseries beta function:
\begin{equation}
\beta^i_{\text{full}}(g, \sigma) = \beta^i_{\text{pert}}(g) + \sum_{n=1}^\infty \sigma^n e^{-nS/g}\beta^{i,(n)}(g)
\end{equation}

\marginnote{A non-perturbative fixed point has $\beta_{\text{pert}} \neq 0$ but the full $\beta_{\text{full}} = 0$ due to cancellation with instanton sectors.}

\subsection{Three Types of Fixed Points}

\textbf{Type 1: Perturbative fixed points} where $\beta_{\text{pert}}(g^*) = 0$ and hence $\beta_{\text{full}}(g^*, 0) = 0$. These are the traditional fixed points visible in perturbation theory. The Gaussian and Wilson-Fisher fixed points are of this type.

\textbf{Type 2: Non-perturbative fixed points} where $\beta_{\text{pert}}(g^*) \neq 0$ but the instanton contributions cancel the perturbative piece:
\begin{equation}
\beta_{\text{pert}}(g^*) + \sum_n (\sigma^*)^n e^{-nS/g^*}\beta^{(n)}(g^*) = 0
\end{equation}
Such fixed points are completely invisible to any finite order of perturbation theory.

\textbf{Type 3: Mixed fixed points} where both perturbative and non-perturbative contributions are essential to the cancellation.

\subsection{Evidence for Non-Perturbative Fixed Points}

Non-perturbative fixed points are not merely theoretical curiosities. There is substantial evidence for their existence in several contexts.

\marginnote{Supersymmetric theories provide the cleanest examples because exact results can be computed using localization.}

In \textbf{supersymmetric gauge theories}, exact results from localization reveal fixed points that cannot be seen perturbatively. The classic example is SQCD, where the Seiberg dual description reveals strong-coupling fixed points.

In \textbf{matrix models}, the large-$N$ expansion can be solved exactly, revealing fixed points at strong coupling where the perturbative expansion around weak coupling gives no hint of their existence.

There is also numerical evidence from \textbf{lattice QCD} suggesting that the theory may have non-perturbative fixed points relevant for the chiral transition.

\begin{workedbox}[Box 4.2: Anatomy of a Non-Perturbative Fixed Point]
\textbf{Toy model:} Consider a one-coupling theory with beta function:
\begin{equation}
\beta_{\text{pert}}(g) = \epsilon g - bg^2 + cg^3 + \cdots
\end{equation}
with $\epsilon, b, c > 0$, and instanton contribution:
\begin{equation}
\beta^{(1)}(g) = d \cdot g^2 e^{-S_0/g}
\end{equation}

\textbf{Perturbative fixed points:} Setting $\beta_{\text{pert}} = 0$ gives:
$g^*_1 = 0$ (Gaussian) and $g^*_2 = \epsilon/b + O(\epsilon^2)$ (like Wilson-Fisher).

\textbf{Full fixed point condition:}
\begin{equation}
\epsilon g^* - b(g^*)^2 + cg^{*3} + \sigma^* d(g^*)^2 e^{-S_0/g^*} = 0
\end{equation}

\textbf{Non-perturbative solution:} If $\sigma^* \neq 0$ and $g^*$ is chosen such that:
\begin{equation}
\sigma^* e^{-S_0/g^*} = \frac{b - \epsilon/g^* - cg^*}{d}
\end{equation}
then we have a fixed point that requires non-zero transseries parameter $\sigma^*$ and is invisible to perturbation theory (the LHS is exponentially small for small $g^*$).

\textbf{Physical interpretation:} The instanton sector provides a ``restoring force'' that balances the perturbative running. The theory sits at an equilibrium between perturbative and non-perturbative effects.
\end{workedbox}

%-------------------------------------------------------------------------------
\section{Stability and Classification}
\label{sec:stability}
%-------------------------------------------------------------------------------

Near any fixed point, perturbations either grow or shrink under RG. This determines the \textbf{universality class}.

\subsection{The Stability Matrix}

Linearize the beta function near a fixed point $g^*$:
\begin{equation}
\frac{d(\delta g^i)}{d\ell} = B^i{}_j \, \delta g^j, \qquad B^i{}_j = \frac{\partial\beta^i}{\partial g^j}\bigg|_{g^*}
\end{equation}

The eigenvalues $\lambda_\alpha$ of the stability matrix $B$ determine the fate of perturbations.

\marginnote{The stability matrix $B$ is the Jacobian of the beta function at the fixed point. Its eigenvalues classify perturbations.}

\subsection{Relevant, Irrelevant, Marginal}

The eigenvectors of $B$ define natural directions in coupling space. Each direction is classified by its eigenvalue.

\textbf{Relevant directions} have $\lambda_\alpha > 0$. Perturbations grow under RG, flowing away from the fixed point. These directions must be tuned to reach the fixed point.

\textbf{Irrelevant directions} have $\lambda_\alpha < 0$. Perturbations shrink under RG, flowing toward the fixed point. These directions are ``self-tuning.''

\textbf{Marginal directions} have $\lambda_\alpha = 0$. The fate depends on higher-order terms.

\begin{workedbox}[Box 4.3: Classification at the Gaussian Fixed Point]
\textbf{Setup:} The 1D $\phi^4$ beta functions are
\begin{align}
\beta_r &= 2r + \frac{3\lambda\Lambda}{\pi(\Lambda^2 + r)} \\
\beta_\lambda &= 2\lambda
\end{align}

\textbf{At the Gaussian} $(r^*, \lambda^*) = (0, 0)$:

The stability matrix is:
\begin{equation}
B = \begin{pmatrix}
\partial\beta_r/\partial r & \partial\beta_r/\partial\lambda \\
\partial\beta_\lambda/\partial r & \partial\beta_\lambda/\partial\lambda
\end{pmatrix}_{(0,0)} = \begin{pmatrix} 2 & 3/(\pi\Lambda) \\ 0 & 2 \end{pmatrix}
\end{equation}

\textbf{Eigenvalues:} Both eigenvalues are $+2$.

\textbf{Classification:} Both directions are \textbf{relevant}. Any perturbation away from $(0,0)$ grows under RG. The Gaussian fixed point is ``completely unstable'' or ``UV attractive.''

\textbf{Interpretation:} To reach the Gaussian fixed point from the IR, we must tune both $r$ and $\lambda$ to zero. There is no basin of attraction.

\textbf{The connection to $\Delta$:} The eigenvalues are the \textbf{scaling dimensions} of the perturbations. Here $\Delta_r = \Delta_\lambda = 2$, matching the engineering dimensions (no anomalous contribution at the Gaussian).
\end{workedbox}

\subsection{Scaling Dimensions and Eigenvalues}

The eigenvalues of $B$ are called \textbf{scaling dimensions} (or ``RG eigenvalues''). They control how perturbations scale:
\begin{equation}
\delta g^\alpha(\ell) \propto e^{\Delta_\alpha \ell}
\end{equation}

\marginnote{Scaling dimensions are ``quantum numbers'' for operators. They determine the power-law behavior of correlation functions.}

A perturbation with dimension $\Delta > 0$ grows (relevant), $\Delta < 0$ shrinks (irrelevant), and $\Delta = 0$ is marginal.

At the Gaussian fixed point, scaling dimensions equal engineering dimensions. At non-trivial fixed points, interactions modify them by the \textbf{anomalous dimension}:
\begin{equation}
\Delta = \Delta_{\text{eng}} + \gamma
\end{equation}

%-------------------------------------------------------------------------------
\section{Universality Classes}
\label{sec:universality}
%-------------------------------------------------------------------------------

Perhaps the most remarkable consequence of the RG is \textbf{universality}: different microscopic theories can exhibit identical macroscopic behavior.

\subsection{The Basin of Attraction}

\marginnote{Universality: water at its critical point and uniaxial magnets at the Curie point are described by the same fixed point and have the same critical exponents.}

The \textbf{basin of attraction} of a fixed point is the set of all theories that flow to it under RG. All theories in the same basin exhibit the same IR behavior. They form a \textbf{universality class}.

Different microscopic theories (lattice models with different interactions, continuum theories with different UV cutoffs) can flow to the same fixed point. Their long-distance behavior is then identical.

\subsection{Why Universality?}

Consider approaching a fixed point along irrelevant directions. By definition, these directions flow toward the fixed point. The ``memory'' of where we started is erased.

Only the relevant directions matter because only they distinguish different theories at long distances. If two theories have the same relevant perturbations tuned in the same way, they approach the same fixed point from the same direction and have identical IR physics.

\subsection{Universality and Transseries}

The concept of universality extends to the full transseries structure. Systems in the same universality class have identical IR behavior not just at the perturbative level but including all non-perturbative sectors.

\marginnote{The Stokes constants are universal quantities characterizing the fixed point, alongside the perturbative critical exponents.}

The \textbf{Stokes constants} are universal. They characterize how the transseries sectors are linked and are independent of the microscopic details. Different systems flowing to the Wilson-Fisher fixed point have the same Stokes constants even if their microscopic instanton actions differ.

This is a strong constraint. It means that the full resurgent structure, not just the leading exponents, is determined by the fixed point.

%-------------------------------------------------------------------------------
\section{The Porous Medium Equation}
\label{sec:pme}
%-------------------------------------------------------------------------------

Our third and final example is the \textbf{porous medium equation} (PME), which governs nonlinear diffusion in porous media. This example exhibits \textbf{anomalous dimensions} that dimensional analysis cannot predict.

\marginnote{The PME describes gas flow through porous rock, groundwater seepage, and heat conduction in plasmas. It's the simplest PDE with anomalous dimensions.}

\subsection{The Model}

The porous medium equation in $d$ dimensions is:
\begin{equation}
\frac{\partial\rho}{\partial t} = D\nabla^2(\rho^m)
\label{eq:pme}
\end{equation}
where $\rho(x, t) \geq 0$ is the density, $D$ is a diffusion coefficient, and $m > 1$ is the nonlinearity exponent.

For $m = 1$, this reduces to the linear heat equation $\partial\rho/\partial t = D\nabla^2\rho$. The nonlinearity $m > 1$ means diffusion is faster where density is higher.

\subsection{Why the PME?}

The PME is ideal for demonstrating anomalous dimensions for several reasons. It's a single PDE with one nonlinearity parameter $m$. Self-similar solutions exist and can be found exactly. Dimensional analysis fails to determine the similarity exponents when $m \neq 1$. And the RG calculation is tractable.

\begin{workedbox}[Box 4.4: Dimensional Analysis for the PME]
\textbf{Setup:} Consider a localized initial condition with total mass $M = \int\rho \, d^dx$. What is the width $L(t)$ at late times?

\textbf{Parameters and dimensions:}
\begin{center}
\begin{tabular}{ccc}
Quantity & Symbol & Dimensions \\
\hline
Width & $L$ & $[L]$ \\
Time & $t$ & $[T]$ \\
Diffusion coefficient & $D$ & $[L^2/T] \cdot [\rho^{1-m}]$ \\
Total mass & $M$ & $[\rho] \cdot [L^d]$ \\
Exponent & $m$ & dimensionless
\end{tabular}
\end{center}

\textbf{For $m = 1$ (linear diffusion):}
$D$ has dimensions $[L^2/T]$. The width must be:
\begin{equation}
L(t) = \sqrt{Dt} \cdot f(M, d)
\end{equation}
For the heat kernel, $f$ is a constant. Result: $L \propto t^{1/2}$ (first-kind self-similarity).

\textbf{For $m \neq 1$:}
$D$ has dimensions that depend on $\rho$, which has no fixed scale! The parameters $D$, $M$, $t$ cannot be combined to give $L$ without knowing how $\rho$ scales.

\textbf{The problem:} Dimensional analysis gives $L \propto t^\alpha$ with $\alpha$ \emph{undetermined}. The exponent must come from solving the equation.
\end{workedbox}

%-------------------------------------------------------------------------------
\section{First-Kind and Second-Kind Self-Similarity}
\label{sec:self_similarity}
%-------------------------------------------------------------------------------

Barenblatt distinguished two types of self-similar solutions, and this distinction is crucial for understanding anomalous dimensions.

\subsection{First-Kind Self-Similarity}

A solution has \textbf{first-kind self-similarity} if dimensional analysis completely determines the scaling exponents.

\marginnote{First-kind: dimensional analysis works. Second-kind: it doesn't. The exponent is ``anomalous.''}

For the linear heat equation ($m = 1$), the fundamental solution is:
\begin{equation}
\rho(x, t) = \frac{1}{(4\pi Dt)^{d/2}}\exp\left(-\frac{|x|^2}{4Dt}\right)
\end{equation}
The width scales as $L \sim t^{1/2}$, exactly as dimensional analysis predicts.

\subsection{Second-Kind Self-Similarity}

A solution has \textbf{second-kind self-similarity} if dimensional analysis fails to determine the exponents. The exponents must be computed dynamically.

For the PME with $m > 1$, the famous Barenblatt-Pattle solution is:
\begin{equation}
\rho(x, t) = \frac{1}{t^\alpha}\left[C - \frac{(m-1)}{4md}\frac{|x|^2}{(Dt)^{2\beta}}\right]_+^{1/(m-1)}
\label{eq:barenblatt}
\end{equation}
where $[y]_+ = \max(y, 0)$ and the exponents satisfy:
\begin{equation}
\alpha = \frac{d}{d(m-1) + 2}, \qquad \beta = \frac{1}{d(m-1) + 2}
\label{eq:barenblatt_exponents}
\end{equation}

\marginnote{The Barenblatt exponents depend on $m$ in a non-trivial way. They are ``anomalous'' in the RG sense.}

The exponents depend on $m$ in a way that dimensional analysis alone cannot predict. The exponent $\beta$ is the \textbf{anomalous dimension} of this problem.

\begin{workedbox}[Box 4.5: Deriving the Barenblatt Exponents]
\textbf{Ansatz:} Seek self-similar solutions of the form:
\begin{equation}
\rho(x, t) = t^{-\alpha}f(\xi), \qquad \xi = \frac{|x|}{t^\beta}
\end{equation}

\textbf{Step 1: Substitute into the PME.}
\begin{equation}
-\alpha t^{-\alpha-1}f - \beta t^{-\alpha-1}\xi f' = Dt^{-m\alpha - 2\beta}\left[\frac{d-1}{\xi}(f^m)' + (f^m)''\right]
\end{equation}

\textbf{Step 2: Require self-consistency.}
For the equation to be satisfied for all $t$, the powers of $t$ must match:
\begin{equation}
-\alpha - 1 = -m\alpha - 2\beta
\end{equation}

\textbf{Step 3: Conservation of mass.}
Total mass $M = \int\rho \, d^dx = t^{-\alpha + d\beta}\int f(\xi)\xi^{d-1}d\xi$ must be constant:
\begin{equation}
-\alpha + d\beta = 0
\end{equation}

\textbf{Step 4: Solve the system.}
From mass conservation: $\alpha = d\beta$.
Substituting into the self-consistency equation:
\begin{equation}
-d\beta - 1 = -md\beta - 2\beta = -\beta(md + 2)
\end{equation}
\begin{equation}
\beta[(md + 2) - d] = 1 \implies \beta = \frac{1}{d(m-1) + 2}
\end{equation}

\textbf{Result:}
\begin{equation}
\boxed{\alpha = \frac{d}{d(m-1)+2}, \qquad \beta = \frac{1}{d(m-1)+2}}
\end{equation}

\textbf{Check:} For $m = 1$: $\beta = 1/2$ (linear diffusion) \checkmark

\textbf{The anomalous dimension:}
Compare to dimensional analysis, which would give $\beta = 1/2$ if it worked. The deviation $\gamma_\beta = \beta - 1/2$ is the anomalous dimension. Its existence signals second-kind self-similarity.
\end{workedbox}

%-------------------------------------------------------------------------------
\section{The PME as an RG Flow}
\label{sec:pme_rg}
%-------------------------------------------------------------------------------

The Barenblatt exponents have a natural interpretation in the RG language. The PME flows to a fixed point where the exponents are determined dynamically.

\subsection{The Parameter Space}

Consider the family of self-similar solutions parameterized by their exponents:
\begin{equation}
\rho_{\alpha,\beta}(x, t) = t^{-\alpha}f_{\alpha,\beta}(|x|/t^\beta)
\end{equation}

Only special values of $(\alpha, \beta)$ give solutions to the PME. The Barenblatt values are a fixed point of the RG in the space of self-similar profiles.

\marginnote{The Barenblatt solution is an RG fixed point in the space of self-similar profiles.}

\subsection{Stability and Selection}

Why does the Barenblatt solution emerge? Other self-similar forms might exist but are unstable. Under the RG (zooming out), generic initial conditions flow toward the stable self-similar profile.

The Barenblatt fixed point is \textbf{IR stable}: perturbations decay as $t \to \infty$. This is why the exponents~\eqref{eq:barenblatt_exponents} are observed experimentally.

\subsection{The Transseries Structure}

The self-similar exponent $\beta$ is computed exactly in this case, but it can also be understood as the sum of a perturbative series plus non-perturbative corrections. Expanding around $m = 1$:
\begin{equation}
\beta = \frac{1}{2} - \frac{d}{4}(m-1) + \frac{d(d+2)}{8}(m-1)^2 - \cdots
\end{equation}

\marginnote{Even for the PME, the exponents can be viewed as transseries with the expansion parameter $(m-1)$.}

This series has Gevrey-1 structure if we consider it as an asymptotic expansion in $(m-1)$. The Borel transform has singularities corresponding to ``instanton'' configurations in the $m$-expansion. While these are less dramatic than QFT instantons, the mathematical structure is identical.

The selection of the physical exponent from among the formal solutions corresponds to a specific resummation prescription. Median resummation gives the real Barenblatt exponent.

%-------------------------------------------------------------------------------
\section{The Wilson-Fisher Fixed Point Revisited}
\label{sec:wilson_fisher}
%-------------------------------------------------------------------------------

The epsilon expansion for the Wilson-Fisher fixed point has a deep resurgent structure that parallels the PME analysis.

\subsection{The Expansion}

The anomalous dimension $\eta$ at the Wilson-Fisher fixed point has the expansion:
\begin{equation}
\eta = \frac{(n+2)}{2(n+8)^2}\epsilon^2 + O(\epsilon^3)
\end{equation}
where $n$ is the number of field components and $\epsilon = 4 - d$.

This series continues to high orders and is known to be asymptotic with factorially growing coefficients.

\marginnote{The epsilon expansion is Gevrey-1. Borel resummation is required for meaningful predictions at $\epsilon = 1$.}

\subsection{Resurgent Analysis}

The Borel transform of the epsilon series has singularities that encode non-perturbative effects. For scalar $\phi^4$ theory:
\begin{equation}
\hat{\eta}(\zeta) \sim \frac{1}{\zeta - \zeta_{\text{ren}}} + \cdots
\end{equation}
where $\zeta_{\text{ren}} = 3/(n+8)$ is the leading renormalon position.

\textbf{Renormalons} are singularities in the Borel plane arising from the factorial growth of perturbative coefficients due to RG running. They are not related to tunneling (unlike instantons) but rather to the inherent ambiguity in defining the perturbative sum at large orders.

\subsection{Physical Predictions}

Despite the divergence, the epsilon expansion gives remarkably accurate predictions. For the 3D Ising model ($n = 1$, $\epsilon = 1$):
\begin{equation}
\eta_{\text{exp}} \approx 0.0363, \qquad \eta_{O(\epsilon^2)} = \frac{3}{242} \approx 0.0124
\end{equation}

Higher-order calculations with resummation give $\eta \approx 0.036$, in excellent agreement with experiment and numerical simulations.

\begin{workedbox}[Box 4.6: Renormalons and the Epsilon Expansion]
\textbf{Why the expansion diverges:}

The $k$-th order contribution to $\eta$ goes like:
\begin{equation}
\eta^{(k)} \sim (-1)^k c \cdot a^k \cdot k! \cdot \epsilon^k
\end{equation}

The factorial $k!$ arises from counting the number of diagrams at high orders and from UV renormalon contributions.

\textbf{Borel transform:}
\begin{equation}
\hat{\eta}(\zeta) = \sum_k \frac{\eta^{(k)}}{k!}\zeta^k \sim \sum_k (-a\zeta)^k = \frac{1}{1 + a\zeta}
\end{equation}
This has a pole at $\zeta = -1/a$, corresponding to a renormalon singularity.

\textbf{Resummation:}
The Borel-Laplace integral:
\begin{equation}
\eta(\epsilon) = \int_0^\infty e^{-\zeta/\epsilon}\hat{\eta}(\zeta)d\zeta
\end{equation}
is ambiguous because the pole lies on the integration contour for real $a < 0$ (alternating signs).

\textbf{Physical prescription:}
Take the principal value or median resummation to get real physical predictions. The ambiguity is of order $e^{-1/(a\epsilon)}$, exponentially small for small $\epsilon$.

\textbf{For $\epsilon = 1$:}
The exponentially small ambiguity becomes order one! Full resurgent analysis is needed to make accurate predictions.
\end{workedbox}

%-------------------------------------------------------------------------------
\section{The Landscape of Fixed Points}
\label{sec:landscape}
%-------------------------------------------------------------------------------

The full picture includes all fixed points, both perturbative and non-perturbative, organized by their stability properties.

\subsection{The RG ``Phase Diagram''}

In the extended parameter space including transseries parameters, fixed points form a landscape. The RG flow connects different fixed points, and the stability structure determines which fixed points are ``reached'' from generic initial conditions.

\marginnote{The full fixed point landscape includes perturbative and non-perturbative fixed points, connected by RG flows in the extended parameter space.}

Generic UV completions flow to IR fixed points. Which IR fixed point is reached depends on the relevant directions and how they are tuned. The irrelevant directions are forgotten.

\subsection{Conformal Windows}

In gauge theories, there can be ranges of parameter space (``conformal windows'') where the theory flows to a non-trivial interacting fixed point rather than to a free theory. The boundaries of these windows are determined by when fixed points collide and disappear.

Non-perturbative fixed points can extend the conformal window beyond what perturbation theory predicts. This is an active area of research in strongly coupled gauge theories.

\subsection{Emergent Symmetry at Fixed Points}

Fixed points often have enhanced symmetry compared to generic points in theory space. Scale invariance is automatic, and under mild conditions scale invariance implies the full conformal symmetry in $d > 2$.

This emergent symmetry provides powerful constraints. Conformal field theory techniques can compute correlation functions exactly at fixed points, even in strongly coupled theories.

%-------------------------------------------------------------------------------
\section{Looking Ahead}
\label{sec:ch4_preview}
%-------------------------------------------------------------------------------

This chapter classified fixed points by stability and introduced anomalous dimensions. The three examples now cover complementary phenomena.

\marginnote{Oscillator: secular terms. $\phi^4$: beta functions. PME: anomalous dimensions. Together they demonstrate the complete RG framework.}

\textbf{The oscillator} demonstrated secular terms and running parameters with trivial fixed point structure. \textbf{The $\phi^4$ theory} showed non-trivial beta functions and the Gaussian fixed point. \textbf{The PME} revealed anomalous dimensions and second-kind self-similarity.

The next two chapters develop the geometric structures underlying parameter space. \textbf{Chapter~\ref{ch:geometry}} introduces metrics on theory space, relating RG to information geometry and gradient flow. \textbf{Chapter~\ref{ch:connections}} develops connections and parallel transport, leading to the unified picture of Stokes phenomena as monodromy.

%-------------------------------------------------------------------------------
\section*{Summary}
\addcontentsline{toc}{section}{Summary}
%-------------------------------------------------------------------------------

\begin{center}
\fbox{\parbox{0.85\textwidth}{
\textbf{Fixed points} satisfy $\beta^i(g^*) = 0$. At fixed points, theories are scale-invariant.

\textbf{Perturbative fixed points} have $\beta_{\text{pert}}(g^*) = 0$. \textbf{Non-perturbative fixed points} have $\beta_{\text{full}}(g^*) = 0$ but $\beta_{\text{pert}}(g^*) \neq 0$.

\textbf{Stability} is determined by the eigenvalues of $B^i{}_j = \partial\beta^i/\partial g^j|_{g^*}$.

\textbf{Classification:}
\begin{itemize}
\item Relevant ($\Delta > 0$): grows under RG
\item Irrelevant ($\Delta < 0$): shrinks under RG
\item Marginal ($\Delta = 0$): fate depends on higher orders
\end{itemize}

\textbf{Universality} means different microscopic theories flow to the same fixed point and have identical IR behavior, including the same Stokes constants.

\textbf{The porous medium equation} $\partial_t\rho = D\nabla^2(\rho^m)$ exhibits \textbf{second-kind self-similarity} with anomalous exponents:
\begin{equation}
\alpha = \frac{d}{d(m-1)+2}, \qquad \beta = \frac{1}{d(m-1)+2}
\end{equation}

\textbf{Anomalous dimensions} are corrections to engineering dimensions from interactions. They signal that dimensional analysis fails and scaling exponents must be computed dynamically.
}}
\end{center}
