%===============================================================================
\chapter{Chaotic Dynamics: The Lorenz System}
\label{ch:lorenz}
%===============================================================================

The Lorenz system provides our first extended application of the RG framework developed in Part I. This chapter demonstrates how the six-step recipe of Chapter~\ref{ch:recipe} applies to chaotic dynamical systems. Originally derived as a simplified model of atmospheric convection, the Lorenz equations exhibit multi-scale behavior that makes them a natural testing ground for RG methods.

We will proceed through the recipe systematically. Step 1 identifies the temporal scale hierarchy created by the parameter $\sigma$, which separates fast velocity relaxation from slow temperature evolution. Step 2 defines coarse-graining through the slow manifold reduction. Step 3 constructs a theory space parametrized by the control parameters $\sigma$, $\rho$, and $\beta$. Step 4 derives the beta functions for slowly varying amplitudes near bifurcation points. Step 5 analyzes the fixed-point landscape, including both the simple fixed points of the flow and the strange attractor viewed as a ``fixed set.''

\marginnote{The Lorenz system was one of the first examples of deterministic chaos, demonstrating that simple equations can produce infinitely complex behavior.}

%-------------------------------------------------------------------------------
\section{The Lorenz Equations}
\label{sec:lorenz_equations}
%-------------------------------------------------------------------------------

The Lorenz system consists of three coupled ordinary differential equations:
\begin{align}
\dot{x} &= \sigma(y - x), \label{eq:lorenz_x}\\
\dot{y} &= x(\rho - z) - y, \label{eq:lorenz_y}\\
\dot{z} &= xy - \beta z. \label{eq:lorenz_z}
\end{align}

Here $x$ represents the intensity of convective motion, $y$ the temperature difference between ascending and descending currents, and $z$ the deviation of the vertical temperature profile from linearity. The parameters are the Prandtl number $\sigma$, the reduced Rayleigh number $\rho$, and a geometric factor $\beta$.

\marginnote{The classic Lorenz values $\sigma = 10$, $\rho = 28$, $\beta = 8/3$ produce the famous butterfly-shaped strange attractor.}

\subsection{Scale Identification}

Following Step 1 of the recipe, we identify the scales in the Lorenz system. The system contains multiple time scales whose separation is controlled by the parameter $\sigma$. This ratio compares the relaxation rates of velocity and temperature: when $\sigma \gg 1$, the variable $x$ adjusts rapidly to $y$, creating a slow manifold on which the effective dynamics is lower-dimensional.

The parameter $\rho$ sets the driving strength and determines the amplitude scales of the motion. Near $\rho = 1$, convection just begins and the motion has small amplitude. As $\rho$ increases, the amplitude of motion grows, and qualitative changes in dynamics occur through bifurcations.

We take $s = \ln t$ as our scale parameter, examining how the effective dynamics changes when we coarse-grain over different time windows. The separation of scales required for RG analysis is present when $\sigma \gg 1$, allowing us to average over the fast relaxation of $x$ to obtain effective equations for the slow variables.

%-------------------------------------------------------------------------------
\section{Fixed Points and Their Stability}
\label{sec:lorenz_fixed}
%-------------------------------------------------------------------------------

The fixed points of the Lorenz system, where $\dot{x} = \dot{y} = \dot{z} = 0$, correspond to steady states of convection.

\subsection{The Origin}

The trivial fixed point $(x^*, y^*, z^*) = (0, 0, 0)$ represents the conductive state with no fluid motion. The stability matrix (equation~\ref{eq:stability_matrix}) at the origin is
\begin{equation}
B = \begin{pmatrix} -\sigma & \sigma & 0 \\ \rho & -1 & 0 \\ 0 & 0 & -\beta \end{pmatrix}.
\end{equation}

The eigenvalues are $\lambda_1 = -\beta$ and
\begin{equation}
\lambda_{2,3} = \frac{-(\sigma + 1) \pm \sqrt{(\sigma + 1)^2 + 4\sigma(\rho - 1)}}{2}.
\end{equation}

\marginnote{The onset of convection at $\rho = 1$ is a pitchfork bifurcation, the fluid analog of a ferromagnetic phase transition.}

For $\rho < 1$, all eigenvalues are negative, and the origin is stable. At $\rho = 1$, one eigenvalue crosses zero, signaling the onset of convection. This is precisely analogous to the relevant operator crossing zero at the Gaussian fixed point in $\phi^4$ theory (Chapter~\ref{ch:fixed_points}).

\subsection{The Convective Fixed Points}

For $\rho > 1$, two symmetric fixed points appear:
\begin{equation}
C^{\pm} = (\pm\sqrt{\beta(\rho-1)}, \pm\sqrt{\beta(\rho-1)}, \rho - 1).
\end{equation}
These represent steady convection rolls rotating in opposite directions.

The stability analysis at $C^{\pm}$ gives eigenvalues that become complex for large enough $\sigma$, indicating oscillatory approach to the fixed point. At the critical value
\begin{equation}
\rho_c = \sigma\frac{\sigma + \beta + 3}{\sigma - \beta - 1}
\end{equation}
a Hopf bifurcation occurs, and the fixed points become unstable.

%-------------------------------------------------------------------------------
\section{The Strange Attractor as a ``Fixed Set''}
\label{sec:strange_attractor}
%-------------------------------------------------------------------------------

For $\rho > \rho_c$, the system exhibits chaotic motion on a strange attractor. While not a fixed point in the traditional sense, the attractor can be viewed as a ``fixed set'' under the RG in an appropriate sense.

\subsection{Invariant Measure}

The strange attractor supports an invariant probability measure $\mu$ that describes the statistical distribution of trajectories in the long-time limit. This measure is the analog of the fixed-point theory in QFT.

\marginnote{The invariant measure on the attractor plays the role of the fixed-point correlation functions in field theory.}

Physical quantities computed with respect to $\mu$ are independent of initial conditions and represent the ``universal'' properties of the chaotic state. The Lyapunov exponents, fractal dimension, and correlation functions are all defined with respect to this measure.

\subsection{Lyapunov Exponents as Scaling Dimensions}

The Lyapunov exponents $\lambda_i$ characterize the rate of separation of nearby trajectories:
\begin{equation}
|\delta \mathbf{x}(t)| \sim e^{\lambda_{\max} t} |\delta \mathbf{x}(0)|.
\end{equation}

These exponents play a role analogous to scaling dimensions at a fixed point. Just as the eigenvalues of the stability matrix classify operators as relevant or irrelevant, the Lyapunov exponents classify perturbations to trajectories. A positive exponent indicates an exponentially growing perturbation, analogous to a relevant operator that drives the system away from a fixed point. A negative exponent indicates an exponentially decaying perturbation, analogous to an irrelevant operator whose effects wash out at long times. A zero exponent is marginal and corresponds to motion along the trajectory itself.

For the Lorenz attractor, there is one positive, one zero, and one negative Lyapunov exponent. The positive exponent is responsible for sensitive dependence on initial conditions---the hallmark of chaos. The zero exponent reflects the continuous time translation symmetry. The negative exponent corresponds to the contraction of phase space volume that makes the attractor a set of measure zero.

%-------------------------------------------------------------------------------
\section{RG for Slow Variables}
\label{sec:lorenz_rg}
%-------------------------------------------------------------------------------

The RG approach to the Lorenz system focuses on the slow variables that emerge when there is a separation of time scales.

\subsection{The Slow Manifold}

When $\sigma \gg 1$, the variable $x$ rapidly equilibrates to $y$. On time scales long compared to $1/\sigma$, the dynamics effectively reduces to the slow manifold defined by $x = y$.

\marginnote{The slow manifold is the invariant manifold that survives coarse-graining over fast time scales.}

To derive the reduced dynamics, we use the multiple scales method. We write
\begin{equation}
x = y + \epsilon x_1 + \epsilon^2 x_2 + \cdots
\end{equation}
where $\epsilon = 1/\sigma$, and require that secular terms vanish.

At leading order, the slow dynamics on the manifold is
\begin{align}
\dot{y} &= y(\rho - 1 - z), \\
\dot{z} &= y^2 - \beta z.
\end{align}

\subsection{Beta Functions for Amplitude}

Following the analysis of the anharmonic oscillator in Chapter~\ref{ch:scale}, we can derive beta functions for slowly varying amplitudes. Near the onset of chaos, the amplitude $A$ of oscillations around the unstable fixed points evolves according to
\begin{equation}
\frac{\dd A}{\dd s} = \lambda A - \kappa A^3 + \cdots
\label{eq:lorenz_amplitude}
\end{equation}
where $\lambda > 0$ near the Hopf bifurcation and $\kappa > 0$ is a nonlinear saturation coefficient.

This amplitude equation has the same structure as the RG flow for $\phi^4$ theory, where the coupling runs as $\dd g/\dd s = -\epsilon g + c g^2$. The fixed point $A^* = \sqrt{\lambda/\kappa}$ corresponds to the limit cycle amplitude.

%-------------------------------------------------------------------------------
\section{Universality and Critical Phenomena}
\label{sec:lorenz_universality}
%-------------------------------------------------------------------------------

The Lorenz system exhibits features analogous to critical phenomena in equilibrium statistical mechanics.

\subsection{Routes to Chaos}

Several universal routes to chaos have been identified, each with its own characteristic scaling behavior. The period-doubling cascade, discovered independently by Feigenbaum and Coullet-Tresser, exhibits universal scaling ratios as successive bifurcations accumulate. Intermittency routes involve alternation between regular and chaotic behavior, with power-law statistics for the duration of regular episodes. The quasiperiodicity route, analyzed by Ruelle, Takens, and Newhouse, shows how three incommensurate frequencies can produce strange attractors.

Each route has universal scaling exponents independent of the specific equations. This universality arises because different systems can flow to the same ``strange attractor fixed point'' under coarse-graining. The specific route determines which fixed point is relevant, but within each universality class the scaling behavior is identical.

\marginnote{Feigenbaum's discovery of universal scaling in period-doubling cascades was one of the great triumphs of the RG approach to dynamical systems.}

\subsection{Feigenbaum Universality}

In the period-doubling route, the parameter values $\rho_n$ where the $n$th period-doubling occurs satisfy
\begin{equation}
\lim_{n \to \infty} \frac{\rho_n - \rho_{n-1}}{\rho_{n+1} - \rho_n} = \delta = 4.669\ldots
\end{equation}
where $\delta$ is the Feigenbaum constant. This number is universal across all unimodal maps undergoing period-doubling.

The RG explanation is that the period-doubling accumulation point is a fixed point of a functional RG transformation, and $\delta$ is determined by the relevant eigenvalue of the linearization around this fixed point.

\subsection{The Chaos--Phase Transition Analogy}

The analogy between the onset of chaos and equilibrium phase transitions runs deep. Table~\ref{tab:chaos_statmech} summarizes the correspondence:

\begin{table}[htbp]
\centering
\caption{Correspondence between equilibrium statistical mechanics and the transition to chaos.}
\label{tab:chaos_statmech}
\begin{tabular}{ll}
\toprule
\textbf{Statistical Mechanics} & \textbf{Dynamical Systems} \\
\midrule
Spatial position $z_j = ja$ & Time $t_j = j\tau$ \\
Spin configuration $[s]$ & Trajectory $[x]$ \\
Boltzmann--Gibbs distribution & Invariant measure $m_\mu$ \\
Dimensionless Hamiltonian $\mathcal{H}$ & Evolution map $f_\mu$ \\
Control parameter $K = \beta J$ & Nonlinearity parameter $\mu$ \\
Correlation length $\xi$ & Characteristic time $\tau$ \\
Thermodynamic limit $N \to \infty$ & Asymptotic dynamics $T \to \infty$ \\
Order parameter $\langle s \rangle$ & Lyapunov exponent $L(\mu)$ \\
Critical point $K_c$ & Onset of chaos $\mu_c$ \\
\bottomrule
\end{tabular}
\end{table}

\marginnote{The Lyapunov exponent $L(\mu)$ serves as the order parameter for chaos: $L > 0$ in the chaotic phase, $L \leq 0$ in the regular phase.}

The Lyapunov exponent $L(\mu)$ plays the role of an order parameter for the transition to chaos. In the regular phase ($\mu < \mu_c$), $L \leq 0$ and nearby trajectories converge or stay bounded. In the chaotic phase ($\mu > \mu_c$), $L > 0$ and trajectories diverge exponentially. Near the transition, the smoothed Lyapunov exponent exhibits power-law scaling:
\begin{equation}
\bar{L}(\mu) \sim (\mu - \mu_c)^\beta \quad \text{for } \mu > \mu_c
\end{equation}
with a universal exponent $\beta$ determined by the specific route to chaos.

\marginnote{Critical slowing down in chaos: the characteristic time $\tau(\mu) \sim |\mu - \mu_c|^{-\nu}$ diverges as the system approaches the transition.}

Just as the correlation length $\xi$ diverges at a thermodynamic critical point, the characteristic time $\tau(\mu)$ diverges at the onset of chaos:
\begin{equation}
\tau(\mu) \sim |\mu - \mu_c|^{-\nu}
\end{equation}
This is the dynamical analog of critical slowing down. Near the transition, the system takes longer and longer to settle into its asymptotic behavior, whether that behavior is periodic or chaotic. The exponent $\nu$ is related to the Feigenbaum constant by $\nu = \ln 2 / \ln \delta$.

This correspondence extends to the RG structure itself. Just as the Ising model's RG acts on spin-block Hamiltonians, the period-doubling RG acts on the space of unimodal maps. The fixed-point map $\varphi^*$ satisfying $R\varphi^* = \varphi^*$ (where $R$ is the doubling transformation) is the analog of the Ising critical Hamiltonian. The universal exponent $\delta$ emerges from the relevant eigenvalue of $DR(\varphi^*)$, exactly paralleling how critical exponents emerge from eigenvalues at the Wilson--Fisher fixed point.

%-------------------------------------------------------------------------------
\section{Connection to the Geometric Framework}
\label{sec:lorenz_geometry}
%-------------------------------------------------------------------------------

We now explicitly connect the Lorenz dynamics to the geometric framework of Part I.

\subsection{Theory Space for Lorenz}

The ``theory space'' for the Lorenz system is the three-dimensional parameter space $(\sigma, \rho, \beta)$. Points in this space correspond to different physical systems (different fluid properties or geometries).

The RG flow on this space describes how the effective parameters change as we coarse-grain. For the amplitude equation~\eqref{eq:lorenz_amplitude}, the flow is
\begin{equation}
\frac{\dd \lambda}{\dd s} = 2\lambda - \text{nonlinear corrections}
\end{equation}
indicating that $\lambda$ is a relevant parameter near the trivial fixed point.

\subsection{Metric from Fluctuations}

By analogy with the Zamolodchikov metric (equation~\ref{eq:zamolodchikov_metric}), we can define a metric on parameter space from the fluctuations in physical observables:
\begin{equation}
G_{\mu\nu} = \langle \delta O_\mu \, \delta O_\nu \rangle
\end{equation}
where $O_\mu$ is the observable conjugate to parameter $\mu$ and the average is over the invariant measure.

This metric encodes the sensitivity of the chaotic dynamics to parameter changes and provides a natural distance function on the space of Lorenz systems.

\subsection{Connection to the Three Canonical Examples}

The Lorenz system connects to each of the three canonical examples from Part I:

\marginnote{Each Part II application maps to one or more canonical examples.}

\textbf{Anharmonic oscillator parallel.} The amplitude equation~\eqref{eq:lorenz_amplitude} has the same structure as the RG equation for the oscillator. Both exhibit secular growth cured by running parameters. The $\lambda^2$ nonlinearity mirrors the $A^2$ dependence in the oscillator's frequency shift.

\textbf{$\phi^4$ theory parallel.} The bifurcation to chaos near $\rho = \rho_c$ resembles the Wilson-Fisher fixed point. The control parameter $(\rho - \rho_c)$ plays the role of the temperature deviation $(T - T_c)$ in critical phenomena. Universal exponents emerge from linearization at the fixed point.

\textbf{PME parallel.} The strange attractor's fractal dimension is an ``anomalous dimension'' not predictable from naive scaling. Like the PME exponents, it must be computed from the dynamics rather than dimensional analysis.

%-------------------------------------------------------------------------------
\section{Summary}
\label{sec:lorenz_summary}
%-------------------------------------------------------------------------------

The Lorenz system demonstrates how the six-step RG recipe applies to chaotic dynamics. We identified the temporal scale hierarchy created by large $\sigma$ (Step 1) and recognized the Gevrey-1 structure of perturbative expansions (Step 2). The theory space is parametrized by the control parameters $\sigma$, $\rho$, and $\beta$, extended to include transseries parameters (Step 3). The beta functions for slowly varying amplitudes, equation~\eqref{eq:lorenz_amplitude}, describe the evolution near bifurcation points (Step 4). The fixed-point analysis (Step 5) revealed both simple fixed points of the flow and the strange attractor as a ``fixed set'' with invariant measure. Physical predictions are extracted via proper resummation (Step 6).

The correspondence between dynamical systems and the RG is systematic. Fixed points of the ODE map to fixed points of the RG flow. The strange attractor supports an invariant measure analogous to a fixed-point theory. Lyapunov exponents play the role of scaling dimensions, classifying perturbations as relevant, irrelevant, or marginal. Slow manifold reduction is the dynamical analog of integrating out irrelevant operators, producing effective equations for the slow variables alone.

The key equations from Part I that we invoked include the stability matrix for fixed point analysis, the envelope method for slow manifold reduction, and the amplitude equation~\eqref{eq:lorenz_amplitude} which takes the same form as the beta function for $\phi^4$ theory. Universal routes to chaos arise from RG fixed points with universal eigenvalues, explaining why systems as different as the Lorenz equations and the logistic map share the same Feigenbaum scaling constants. This application demonstrates that the geometric RG framework extends naturally beyond equilibrium statistical mechanics to dissipative dynamical systems.

