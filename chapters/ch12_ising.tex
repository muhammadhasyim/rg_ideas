%===============================================================================
\chapter{Conformal Field Theory: The 2D Ising Model}
\label{ch:ising}
%===============================================================================

The two-dimensional Ising model holds a special place in the history of physics as one of the few exactly solvable interacting systems. At its critical point, it exhibits conformal invariance and provides a beautiful testing ground for the RG framework. This chapter applies the six-step recipe of Chapter~\ref{ch:recipe} from multiple perspectives, demonstrating how the same physics emerges whether we use real-space blocking, field theory, conformal methods, or free fermion techniques.

The scale hierarchy (Step 1) ranges from the lattice spacing $a$ to the diverging correlation length $\xi$. Coarse-graining (Step 2) can be implemented via Kadanoff's block-spin transformation, momentum-shell integration, or operator product expansion. Theory space (Step 3) has coordinates $(K, H)$ representing temperature and magnetic field. The beta functions (Step 4) follow exactly from conformal invariance at the fixed point. Fixed-point analysis (Step 5) reveals the critical point with its exact scaling dimensions $\Delta_\sigma = 1/16$ and $\Delta_\varepsilon = 1/2$.

\marginnote{The 2D Ising model was solved exactly by Onsager in 1944, one of the great achievements of theoretical physics.}

%-------------------------------------------------------------------------------
\section{The Lattice Model}
\label{sec:ising_lattice}
%-------------------------------------------------------------------------------

Consider a square lattice with spin variables $\sigma_i = \pm 1$ at each site. The Hamiltonian is
\begin{equation}
H = -J \sum_{\langle i,j \rangle} \sigma_i \sigma_j - h \sum_i \sigma_i
\label{eq:ising_hamiltonian}
\end{equation}
where $J > 0$ is the ferromagnetic coupling, $h$ is an external magnetic field, and $\langle i,j \rangle$ denotes nearest neighbors.

\subsection{Scale Identification}

Following Step 1 of the recipe, we identify the scales. The lattice spacing $a$ provides the UV cutoff, below which the discrete nature of the spins matters. The correlation length $\xi$ provides the IR scale, diverging at the critical point as $\xi \sim |T - T_c|^{-\nu}$ with the critical exponent $\nu = 1$.

Temperature is the control parameter. At $T = T_c$ and external field $h = 0$, the system sits at the critical point where fluctuations occur on all length scales from $a$ to $\xi = \infty$. The dimensionless couplings are $K = J/(k_B T)$ and $H = h/(k_B T)$, which parametrize the theory space for this model.

%-------------------------------------------------------------------------------
\section{Kadanoff's Real-Space RG}
\label{sec:kadanoff}
%-------------------------------------------------------------------------------

The conceptually clearest approach to the RG in the Ising model is Kadanoff's block-spin transformation.

\subsection{Block Spin Construction}

Divide the lattice into blocks of $b \times b$ spins. Define a new ``block spin'' $\sigma'_I$ for each block $I$ using a majority rule:
\begin{equation}
\sigma'_I = \text{sign}\left( \sum_{i \in I} \sigma_i \right).
\end{equation}

\marginnote{Kadanoff's blocking procedure makes the RG coarse-graining physically transparent.}

After this transformation, the lattice has spacing $a' = ba$, and we have integrated out fluctuations at scales smaller than $ba$.

\subsection{The RG Transformation}

To preserve the partition function, the block spins must interact with effective couplings $(K', H')$ determined by
\begin{equation}
Z(K, H; N) = Z(K', H'; N/b^2)
\end{equation}
where $N$ is the number of spins.

For the 2D Ising model, the RG transformation takes the form
\begin{align}
K' &= R_K(K, H), \\
H' &= R_H(K, H).
\end{align}

\subsection{Fixed Points and Critical Behavior}

The critical point corresponds to a fixed point $(K^*, H^*)$ where
\begin{equation}
K^* = R_K(K^*, 0), \quad H^* = 0.
\end{equation}

Linearizing around the fixed point:
\begin{equation}
\begin{pmatrix} \delta K' \\ \delta H' \end{pmatrix} = \begin{pmatrix} \frac{\partial R_K}{\partial K} & \frac{\partial R_K}{\partial H} \\ \frac{\partial R_H}{\partial K} & \frac{\partial R_H}{\partial H} \end{pmatrix}_{K^*,0} \begin{pmatrix} \delta K \\ \delta H \end{pmatrix}.
\end{equation}

The eigenvalues $\lambda_t$ and $\lambda_h$ of this matrix give the critical exponents via
\begin{equation}
\nu = \frac{\ln b}{\ln \lambda_t}, \quad \Delta_h = \frac{\ln \lambda_h}{\ln b}
\end{equation}
where $\Delta_h$ is the scaling dimension of the magnetic field operator.

\marginnote{The critical exponents are eigenvalues of the linearized RG, exactly as developed in Chapter~\ref{ch:fixed_points}.}

%-------------------------------------------------------------------------------
\section{The $\phi^4$ Field Theory Approach}
\label{sec:ising_phi4}
%-------------------------------------------------------------------------------

In the continuum limit, the Ising model is described by a scalar field theory.

\subsection{Continuum Limit}

Near the critical point, the lattice model can be replaced by a continuum action:
\begin{equation}
S[\phi] = \int d^2 x \left[ \frac{1}{2}(\nabla \phi)^2 + \frac{r}{2}\phi^2 + \frac{u}{4!}\phi^4 \right]
\label{eq:ising_action}
\end{equation}
where $\phi(x)$ is a coarse-grained magnetization field.

This is the O(1) case of the O(N) model studied in Chapter~\ref{ch:on_model}. The critical point corresponds to the Wilson-Fisher fixed point (which becomes nontrivial in $d < 4$).

\subsection{Two-Dimensional Peculiarities}

In $d = 2$, the $\phi^4$ theory is super-renormalizable. The coupling $u$ has dimension $[u] = 4 - d = 2$, making it strongly relevant. The RG flow drives the system rapidly away from the Gaussian fixed point toward a strongly coupled fixed point.

This is where exact methods and conformal field theory become essential.

%-------------------------------------------------------------------------------
\section{The CFT Approach}
\label{sec:ising_cft}
%-------------------------------------------------------------------------------

At the critical point, the 2D Ising model possesses full conformal invariance, not just scale invariance.

\subsection{Conformal Symmetry}

In two dimensions, the conformal group is infinite-dimensional. Holomorphic coordinate transformations $z \to f(z)$ and antiholomorphic $\bar{z} \to \bar{f}(\bar{z})$ form two copies of the Virasoro algebra with generators $L_n$ and $\bar{L}_n$ satisfying:
\begin{equation}
[L_m, L_n] = (m-n)L_{m+n} + \frac{c}{12}m(m^2-1)\delta_{m+n,0}.
\label{eq:virasoro}
\end{equation}

\marginnote{The central charge $c$ is the most important invariant of a 2D CFT, encoding the number of degrees of freedom.}

The central charge $c$ appears in the anomalous term. For the Ising CFT:
\begin{equation}
c = \frac{1}{2}.
\end{equation}

\subsection{Primary Operators}

The spectrum of the Ising CFT consists of primary operators $\mathcal{O}_\Delta$ labeled by their scaling dimensions $\Delta$. The identity operator $\mathbb{1}$ has $\Delta = 0$, as required by conformal invariance. The spin field $\sigma$ represents the local magnetization and has the non-trivial scaling dimension $\Delta = 1/16$. The energy density $\varepsilon$ measures the local deviation from criticality and has scaling dimension $\Delta = 1/2$. These dimensions are exact, determined by the representation theory of the Virasoro algebra, not by perturbative calculations.

\subsection{Connection to RG}

The CFT perspective provides a complete solution to the RG at the fixed point. Scaling dimensions are eigenvalues of the dilation operator $D = L_0 + \bar{L}_0$, which generates scale transformations in the conformal algebra. Correlation functions are completely determined by conformal symmetry up to a finite number of constants, the OPE coefficients. The operator product expansion provides the connection structure discussed in Chapter~\ref{ch:connections}, relating operators at different points in spacetime.

The correlation length exponent can be read off directly from the scaling dimension of the energy operator:
\begin{equation}
\nu = \frac{1}{2 - \Delta_\varepsilon} = \frac{1}{2 - 1/2} = 1.
\end{equation}
This exact result confirms that the 2D Ising model is in a universality class distinct from mean field theory, which predicts $\nu = 1/2$.

\marginnote{The exact exponents from CFT confirm that 2D Ising is in a universality class distinct from mean field theory.}

%-------------------------------------------------------------------------------
\section{Grassmann Variables and Free Fermions}
\label{sec:ising_grassmann}
%-------------------------------------------------------------------------------

A remarkable feature of the 2D Ising model is its equivalence to a theory of free fermions.

\subsection{The Transfer Matrix}

The partition function can be written as
\begin{equation}
Z = \mathrm{Tr} \, T^M
\end{equation}
where $T$ is the transfer matrix acting on a row of spins and $M$ is the number of rows.

The transfer matrix can be diagonalized using a Jordan-Wigner transformation to fermionic variables.

\subsection{Grassmann Representation}

Define Grassmann (anticommuting) variables $\psi_i$, $\bar{\psi}_i$ satisfying $\{\psi_i, \psi_j\} = \{\bar{\psi}_i, \bar{\psi}_j\} = \{\psi_i, \bar{\psi}_j\} = 0$.

The partition function becomes a Gaussian integral over Grassmann variables:
\begin{equation}
Z = \int \prod_i d\bar{\psi}_i d\psi_i \, e^{-S_F}
\end{equation}
with the free fermion action
\begin{equation}
S_F = \sum_{\langle i,j \rangle} \bar{\psi}_i M_{ij} \psi_j.
\label{eq:ising_fermion}
\end{equation}

\marginnote{The mapping to free fermions makes the Ising model exactly solvable, but the same technique does not generalize to higher dimensions.}

\subsection{Relation to CFT}

The continuum limit of the free fermion theory is a CFT with $c = 1/2$. The spin field $\sigma$ is not part of the free fermion theory but can be constructed as a ``disorder operator'' that creates a branch cut in the fermion propagator.

This construction explains why $\Delta_\sigma = 1/16$ is not a simple multiple of the fermion dimension.

%-------------------------------------------------------------------------------
\section{RG Near the Critical Point}
\label{sec:ising_rg_flow}
%-------------------------------------------------------------------------------

Away from criticality, the RG flow describes how the Ising model approaches or departs from the critical fixed point.

\subsection{Relevant Perturbations}

The critical theory has two relevant perturbations corresponding to the two relevant directions in the linearized RG. Temperature deviation adds a term $\delta \mathcal{L} \sim t \, \varepsilon(x)$ to the Lagrangian, with $t \propto T - T_c$ measuring the distance from criticality. A magnetic field adds $\delta \mathcal{L} \sim h \, \sigma(x)$, breaking the $\mathbb{Z}_2$ symmetry. Both perturbations are relevant because $\Delta_\varepsilon = 1/2$ and $\Delta_\sigma = 1/16$ are both less than the spatial dimension $d = 2$.

\subsection{Beta Functions}

Near the critical point, the beta functions for the dimensionless couplings are:
\begin{align}
\beta_t &= (2 - \Delta_\varepsilon) t = \frac{3}{2} t, \\
\beta_h &= (2 - \Delta_\sigma) h = \frac{15}{8} h.
\end{align}

These are determined exactly by the scaling dimensions.

\marginnote{The exact beta functions confirm the structure derived in Chapter~\ref{ch:rg_equation}.}

%-------------------------------------------------------------------------------
\section{The Zamolodchikov Metric and c-Theorem}
\label{sec:ising_metric}
%-------------------------------------------------------------------------------

The 2D Ising model provides a concrete example of the geometric structures in Part I.

\subsection{The Metric on Theory Space}

Following Chapter~\ref{ch:geometry}, the Zamolodchikov metric on the $(t, h)$ coupling space is:
\begin{equation}
G_{ij} = \int d^2 x \, |x|^4 \langle \mathcal{O}_i(x) \mathcal{O}_j(0) \rangle
\end{equation}
where $\mathcal{O}_1 = \varepsilon$ and $\mathcal{O}_2 = \sigma$.

At the critical point, conformal invariance completely determines the two-point functions:
\begin{equation}
\langle \varepsilon(x) \varepsilon(0) \rangle = \frac{C_\varepsilon}{|x|^{2\Delta_\varepsilon}} = \frac{C_\varepsilon}{|x|}
\end{equation}
and similarly for $\sigma$.

\subsection{The c-Function}

The Zamolodchikov c-function interpolates between fixed points:
\begin{equation}
C(t, h) = c_{\text{UV}} - \text{(positive contribution from flow)}
\end{equation}

\marginnote{The c-theorem ensures that $c$ decreases monotonically along any RG trajectory.}

For flows in the $(t, h)$ plane, $C$ decreases from its value at the Ising fixed point ($c = 1/2$) to zero in the ordered or disordered phases.

%-------------------------------------------------------------------------------
\section{Comparing the Four Approaches}
\label{sec:ising_comparison}
%-------------------------------------------------------------------------------

The 2D Ising model admits four distinct but equivalent treatments, each illuminating different aspects of the physics. The fact that all four give identical physical predictions is a powerful consistency check on the RG framework.

\textbf{Kadanoff real-space RG}: This approach directly implements coarse-graining on the lattice by grouping spins into blocks and defining new effective spins for each block. The method provides an intuitive geometric picture of the RG transformation and gives approximate critical exponents that become exact in certain limits or with improved blocking schemes. Its main limitation is the accumulation of errors from truncating the growing number of couplings generated at each step.

\textbf{$\phi^4$ field theory}: The continuum limit replaces the discrete spin variable with a continuous field $\phi(x)$, connecting to the general framework of Chapter~\ref{ch:on_model}. In two dimensions, the field theory is strongly coupled (since $\varepsilon = 4 - d = 2$ is large), making perturbation theory less reliable than in higher dimensions. Nevertheless, the $\phi^4$ formulation provides the natural bridge to quantum field theory methods and relates the Ising model to the universal O(1) symmetry class.

\textbf{CFT}: At the critical point, the 2D Ising model becomes a conformal field theory with central charge $c = 1/2$. The infinite-dimensional Virasoro symmetry completely determines all scaling dimensions and correlation functions, providing exact non-perturbative results. The CFT approach is most powerful for two-dimensional systems where conformal symmetry is especially constraining; it identifies the Ising CFT as the first in the discrete series of minimal models.

\textbf{Grassmann/fermion}: The mapping to free fermions, possible only in two dimensions, makes the model exactly solvable via standard quadratic path integral methods. This provides a rigorous benchmark for comparing approximate methods and demonstrates that the critical behavior emerges from the massless fermion dispersion relation. The fermion representation also reveals the topological structure underlying the model and connects to modern developments in fermionic topological phases.

\marginnote{The equivalence of these four approaches---lattice, field theory, CFT, and free fermions---is a deep manifestation of universality.}

All four approaches give the same physical predictions for critical exponents, correlation functions, and thermodynamic quantities. This remarkable consistency demonstrates both the universality of critical phenomena and the internal coherence of the RG framework.

%-------------------------------------------------------------------------------
\section{Connection to the Geometric Framework}
\label{sec:ising_geometry}
%-------------------------------------------------------------------------------

The Ising model illustrates every aspect of Part I:

\subsection{Chapter Connections}

\textbf{Scale and Dilation} (Chapters~\ref{ch:scale},~\ref{ch:flows}): The lattice spacing provides the UV scale; the correlation length provides the IR scale. Scaling dimensions classify operators.

\textbf{RG Equation} (Chapter~\ref{ch:rg_equation}): The Kadanoff transformation directly implements the RG. Beta functions are determined by scaling dimensions.

\textbf{Fixed Points} (Chapter~\ref{ch:fixed_points}): The critical point is a fixed point of the RG. Eigenvalues of the linearization give critical exponents.

\textbf{Theory Space} (Chapter~\ref{ch:geometry}): The $(t, h)$ plane is the theory space. The Zamolodchikov metric gives it Riemannian structure.

\textbf{Irreversibility}: The c-theorem ensures irreversibility with $c = 1/2$ at the Ising fixed point.

\textbf{Connections} (Chapter~\ref{ch:connections}): The OPE provides the connection structure, relating operators at different points.

\subsection{Connection to the Three Canonical Examples}

The Ising model connects to each of the three canonical examples:

\marginnote{The 2D Ising model is the canonical example for critical phenomena, providing exact benchmarks for all RG concepts.}

\textbf{Anharmonic oscillator parallel.} While the oscillator has no non-trivial fixed points, the mechanism is the same: parameters that appear constant become scale-dependent. In the Ising model, the temperature coupling runs to zero or infinity depending on starting point.

\textbf{$\phi^4$ theory parallel.} The Ising model \emph{is} the $\phi^4$ theory in the limit $d \to 2$. The Wilson-Fisher fixed point in $d = 4 - \epsilon$ connects continuously to the exact Ising fixed point as $\epsilon \to 2$. Critical exponents computed via the $\epsilon$-expansion can be checked against exact values.

\textbf{PME parallel.} The exact anomalous dimensions $\eta = 1/4$, $\Delta_\sigma = 1/16$ demonstrate second-kind self-similarity. These cannot be guessed from dimensional analysis but emerge from the dynamical equations (here, conformal bootstrap constraints). The Ising model is the exactly solvable limit of anomalous scaling.

%-------------------------------------------------------------------------------
\section{Summary}
\label{sec:ising_summary}
%-------------------------------------------------------------------------------

The 2D Ising model demonstrates the six-step RG recipe in an exactly solvable setting. The scale hierarchy (Step 1) extends from lattice spacing to correlation length. Perturbative analysis and Borel structure (Step 2) can be verified against exact results. Theory space (Step 3) is the two-dimensional $(t, h)$ plane, extended to include non-perturbative sectors. The beta functions (Step 4) are determined exactly by conformal invariance. Fixed-point analysis (Step 5) reveals the critical point with central charge $c = 1/2$ and exact scaling dimensions. Physical predictions (Step 6) are exact here, providing benchmarks for resummation methods.

The power of the Ising model as a testing ground lies in the consistency of all approaches. Real-space, field theory, CFT, and fermion methods all give identical results, confirming the universality of the RG framework. The exact scaling dimensions $\Delta_\sigma = 1/16$ and $\Delta_\varepsilon = 1/2$ provide non-trivial tests: these are not simple fractions but emerge from the representation theory of the Virasoro algebra. The c-theorem is realized explicitly, with the central charge $c = 1/2$ at the critical point decreasing to $c = 0$ in the ordered or disordered phases.

The geometric framework of Part I achieves exact realization in this celebrated system. The stability matrix eigenvalues give critical exponents. The Zamolodchikov metric can be computed from two-point functions. The gradient flow structure and c-theorem hold exactly. The OPE provides the connection structure of Chapter~\ref{ch:connections}. The Ising model demonstrates that the abstract geometric RG framework produces concrete, exact predictions when applied to systems with sufficient symmetry.

