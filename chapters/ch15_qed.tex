%===============================================================================
\chapter{Quantum Electrodynamics}
\label{ch:qed}
%===============================================================================

Quantum electrodynamics (QED) is the relativistic quantum field theory of the electromagnetic interaction, and it was here that renormalization was first developed as a systematic procedure. This chapter applies the five-step recipe of Chapter~\ref{ch:recipe} to QED, showing how the abstract geometric framework of Part I manifests in the physical phenomenon of charge screening. The extraordinary agreement between QED predictions and experiment provides the most precise test of quantum field theory.

The scale hierarchy (Step 1) ranges from the electron mass $m$ (the IR scale) through the renormalization scale $\mu$ to the UV cutoff $\Lambda$. Coarse-graining (Step 2) integrates out high-momentum modes, generating effective couplings. Theory space (Step 3) is parametrized by the fine structure constant $\alpha$ and the electron mass $m$. The beta function (Step 4) is computed from vacuum polarization, yielding $\beta_\alpha = 2\alpha^2/(3\pi) + O(\alpha^3)$. Fixed-point analysis (Step 5) reveals that $\alpha^* = 0$ is an IR-stable fixed point, explaining why electromagnetism appears weakly coupled at everyday energies.

\marginnote{QED achieved unprecedented agreement between theory and experiment, with the electron magnetic moment predicted to better than one part in a trillion.}

%-------------------------------------------------------------------------------
\section{The QED Lagrangian}
\label{sec:qed_lagrangian}
%-------------------------------------------------------------------------------

The QED Lagrangian density is
\begin{equation}
\mathcal{L} = \bar{\psi}(i\gamma^\mu D_\mu - m)\psi - \frac{1}{4}F_{\mu\nu}F^{\mu\nu}
\label{eq:qed_lagrangian}
\end{equation}
where $\psi$ is the electron field, $A_\mu$ the photon field, $D_\mu = \partial_\mu + ieA_\mu$ the covariant derivative, $F_{\mu\nu} = \partial_\mu A_\nu - \partial_\nu A_\mu$ the field strength, and $m$ the electron mass.

\subsection{Scale Identification}

Following Step 1 of the recipe, we identify the scales. The energy scale $\mu$ characterizes the typical momentum transfer in a scattering process; this is the scale at which we probe the electromagnetic interaction. The electron mass $m \approx 0.511$ MeV provides an IR scale below which electron-positron pairs cannot be created, setting a threshold for vacuum polarization effects.

The cutoff $\Lambda$ is the UV scale where the effective field theory description breaks down and new physics must enter. In practice, QED is embedded in the electroweak theory at scales of order 100 GeV. The dimensionless coupling is the fine structure constant $\alpha = e^2/(4\pi) \approx 1/137$, whose small value makes perturbation theory extraordinarily successful.

%-------------------------------------------------------------------------------
\section{Canonical Scaling Dimensions}
\label{sec:qed_canonical}
%-------------------------------------------------------------------------------

Following the analysis of Chapter~\ref{ch:flows}, we determine the canonical dimensions from the Lagrangian.

In $d = 4$ dimensions:
\begin{equation}
[\psi] = \frac{3}{2}, \quad [A_\mu] = 1, \quad [e] = 0, \quad [m] = 1.
\end{equation}

\marginnote{The dimensionlessness of $e$ in $d=4$ makes QED marginal at the classical level, with quantum corrections determining its fate.}

The charge $e$ is classically dimensionless, indicating that the interaction is marginal. Quantum corrections will determine whether the coupling is marginally relevant or irrelevant.

%-------------------------------------------------------------------------------
\section{Running of the Coupling}
\label{sec:qed_running}
%-------------------------------------------------------------------------------

The beta function describes how the effective coupling changes with energy scale.

\subsection{Vacuum Polarization}

The photon propagator receives quantum corrections from virtual electron-positron pairs. These corrections are summarized by the vacuum polarization tensor $\Pi_{\mu\nu}(q)$:
\begin{equation}
\Pi_{\mu\nu}(q) = (q^2 g_{\mu\nu} - q_\mu q_\nu)\Pi(q^2).
\end{equation}

The function $\Pi(q^2)$ contains a logarithmic dependence on momentum:
\begin{equation}
\Pi(q^2) = -\frac{\alpha}{3\pi}\ln\frac{q^2}{m^2} + \text{finite terms}
\end{equation}
for $|q^2| \gg m^2$.

\marginnote{Vacuum polarization represents the ``dressing'' of the photon by virtual particles, screening the bare charge.}

\subsection{The QED Beta Function}

The beta function for $\alpha$ is obtained from the RG equation (Chapter~\ref{ch:rg_equation}):
\begin{equation}
\beta_\alpha \equiv \mu \frac{d\alpha}{d\mu} = \frac{2\alpha^2}{3\pi} + O(\alpha^3).
\label{eq:qed_beta}
\end{equation}

This positive beta function indicates that $\alpha$ increases with energy scale (UV) and decreases toward lower energies (IR).

\subsection{Physical Interpretation}

The running coupling can be integrated:
\begin{equation}
\alpha(\mu) = \frac{\alpha(m)}{1 - \frac{2\alpha(m)}{3\pi}\ln(\mu/m)}.
\label{eq:qed_running}
\end{equation}

At low energies $\mu \ll m$, virtual pairs cannot be created, and $\alpha$ approaches its observed value $\alpha \approx 1/137$. At high energies, the coupling increases due to charge screening by virtual pairs.

%-------------------------------------------------------------------------------
\section{Fixed Points and the Landau Pole}
\label{sec:qed_fixed}
%-------------------------------------------------------------------------------

Applying the framework of Chapter~\ref{ch:fixed_points} to QED reveals important features.

\subsection{The Gaussian Fixed Point}

The only perturbatively accessible fixed point is $\alpha^* = 0$ (the Gaussian or free theory). At this fixed point:
\begin{equation}
\frac{\partial \beta_\alpha}{\partial \alpha}\Big|_{\alpha=0} = 0.
\end{equation}

The coupling is marginal at leading order. The positive coefficient in~\eqref{eq:qed_beta} makes it marginally irrelevant in the IR: the theory flows toward the free fixed point at low energies.

\marginnote{The Gaussian fixed point is an IR attractor for QED, explaining why electromagnetism appears weakly coupled at everyday energies.}

\subsection{The Landau Pole}

From equation~\eqref{eq:qed_running}, the coupling diverges at the ``Landau pole'':
\begin{equation}
\Lambda_{\text{Landau}} = m \exp\left(\frac{3\pi}{2\alpha(m)}\right) \sim 10^{286} \text{ GeV}.
\label{eq:landau_pole}
\end{equation}

This enormously high scale is far beyond any accessible energy, but the existence of the Landau pole indicates that QED cannot be a complete theory valid at all energies.

\subsection{UV Incompleteness and Resurgent Self-Completion}

The Landau pole suggests that QED requires a UV completion---but there are two qualitatively different possibilities.

\textbf{Wilsonian completion.} New physics takes over before the coupling becomes strong. In the Standard Model, QED is embedded in the electroweak theory at scale $\sim 100$ GeV, far below the Landau pole.

\textbf{Non-Wilsonian completion via resurgence.} The transseries framework of Chapter~\ref{ch:rg_equation} offers an alternative. The perturbative beta function $\beta_\alpha = 2\alpha^2/(3\pi) + O(\alpha^3)$ is asymptotic, with renormalon singularities in its Borel transform at positions:
\begin{equation}
\zeta_k = \frac{3\pi k}{2}, \qquad k = 1, 2, 3, \ldots
\end{equation}

\marginnote{The Landau pole may be an artifact of perturbation theory. The resurgent completion could reveal a UV fixed point, rendering QED self-consistent without new particles.}

When these renormalon contributions are included, the full transseries beta function becomes:
\begin{equation}
\beta_{\text{full}}(\alpha) = \beta_{\text{pert}}(\alpha) + \sum_k C_k \, e^{-3\pi k/(2\alpha)} R_k(\alpha)
\end{equation}

For certain values of the transseries parameters $C_k$, this effective beta function can vanish at a finite coupling $\alpha^*_{\text{NP}}$---a \textbf{non-perturbative fixed point}. At this fixed point, QED becomes scale-invariant without the coupling diverging.

Whether such a fixed point exists in QED is an open question. If it does, it would represent \textbf{non-Wilsonian UV completion}: the theory is consistent to arbitrarily high energies without requiring new heavy particles. The Landau pole would be an artifact of truncating the transseries, not a genuine inconsistency.

This connects to the broader theme of Chapter~\ref{ch:fixed_points}: fixed points may exist beyond perturbation theory, visible only through the resurgent structure of the RG equation.

%-------------------------------------------------------------------------------
\section{Anomalous Dimensions}
\label{sec:qed_anomalous}
%-------------------------------------------------------------------------------

Beyond the running of $\alpha$, quantum corrections also modify the scaling dimensions of operators.

\subsection{Electron Field Anomalous Dimension}

The electron propagator receives corrections that modify its scaling behavior. The anomalous dimension $\gamma_\psi$ is defined through:
\begin{equation}
\mu \frac{d}{d\mu} \psi_R = \gamma_\psi \psi_R
\end{equation}
where $\psi_R$ is the renormalized field.

At one loop in QED:
\begin{equation}
\gamma_\psi = \frac{\alpha}{4\pi}(3 - \xi)
\end{equation}
where $\xi$ is the gauge parameter. In Landau gauge ($\xi = 0$), $\gamma_\psi = \frac{3\alpha}{4\pi}$.

\marginnote{The anomalous dimension represents the deviation from classical scaling due to quantum fluctuations.}

\subsection{Electron Mass Running}

The electron mass also runs with scale:
\begin{equation}
\mu \frac{dm}{d\mu} = -\frac{3\alpha}{\pi} m + O(\alpha^2).
\end{equation}

This indicates that the electron mass is a relevant perturbation: at the IR fixed point ($\alpha = 0$), massive electrons decouple from massless photons.

%-------------------------------------------------------------------------------
\section{Ward Identities and Gauge Invariance}
\label{sec:qed_ward}
%-------------------------------------------------------------------------------

Gauge invariance imposes powerful constraints on the RG flow.

\subsection{The Ward-Takahashi Identity}

The conservation of the electromagnetic current implies:
\begin{equation}
q^\mu \Gamma_\mu(p', p) = e[S^{-1}(p') - S^{-1}(p)]
\end{equation}
where $\Gamma_\mu$ is the vertex function and $S$ is the electron propagator.

This identity ensures that the charge renormalization $Z_1$ equals the electron field renormalization $Z_2$:
\begin{equation}
Z_1 = Z_2.
\end{equation}

\marginnote{Ward identities are consequences of gauge symmetry that constrain the form of quantum corrections.}

\subsection{Consequences for the RG}

The Ward identity has profound consequences for the structure of the RG in QED. The electric charge is universally defined and scheme-independent, a remarkable constraint that does not hold for arbitrary couplings. The beta function~\eqref{eq:qed_beta} is gauge invariant, taking the same form in any gauge. Only one independent renormalization constant---$Z_3$ for the photon field---determines the running of $\alpha$, dramatically simplifying the structure of theory space. This exemplifies how symmetries constrain the geometry of theory space, as discussed abstractly in Chapter~\ref{ch:connections}.

%-------------------------------------------------------------------------------
\section{The Lamb Shift and Anomalous Magnetic Moment}
\label{sec:qed_precision}
%-------------------------------------------------------------------------------

The RG framework in QED makes precise predictions that have been verified experimentally.

\subsection{The Lamb Shift}

The energy levels of hydrogen receive corrections from vacuum polarization and electron self-energy. The splitting between the $2S_{1/2}$ and $2P_{1/2}$ states (which are degenerate in the Dirac theory) is:
\begin{equation}
\Delta E_{\text{Lamb}} \approx \frac{\alpha^5 m_e c^2}{6\pi}\left[\ln\frac{1}{\alpha^2} - \ln 2 + \frac{5}{6}\right] \approx 1057 \text{ MHz}.
\end{equation}

This prediction agrees with experiment to high precision.

\subsection{Anomalous Magnetic Moment}

The electron magnetic moment is predicted to differ from the Dirac value $g = 2$:
\begin{equation}
a_e \equiv \frac{g-2}{2} = \frac{\alpha}{2\pi} + O(\alpha^2).
\end{equation}

\marginnote{The anomalous magnetic moment of the electron is the most precisely tested prediction in all of physics.}

Including higher-order corrections (which require sophisticated RG techniques), the theoretical prediction agrees with the experimental value to better than one part in $10^{12}$.

%-------------------------------------------------------------------------------
\section{QED in External Fields}
\label{sec:qed_external}
%-------------------------------------------------------------------------------

QED in strong external fields provides another arena for RG methods.

\subsection{Schwinger Effect}

In a constant electric field $E$, virtual electron-positron pairs can become real if the field is strong enough. The pair production rate per unit volume is:
\begin{equation}
w = \frac{\alpha E^2}{\pi^2} \sum_{n=1}^\infty \frac{1}{n^2} e^{-\frac{\pi m^2 n}{eE}}.
\end{equation}

This nonperturbative effect lies beyond the perturbative RG but can be understood through instanton methods.

\subsection{Euler-Heisenberg Effective Action}

At energies below the electron mass, the physics is described by an effective action for the electromagnetic field alone:
\begin{equation}
\mathcal{L}_{\text{eff}} = -\frac{1}{4}F^2 + \frac{\alpha^2}{90m^4}\left[(F^2)^2 + \frac{7}{4}(F\tilde{F})^2\right] + \cdots
\end{equation}
where $\tilde{F}^{\mu\nu} = \frac{1}{2}\epsilon^{\mu\nu\rho\sigma}F_{\rho\sigma}$ is the dual field strength.

This is the effective theory obtained after integrating out electrons, implementing the RG procedure discussed in Chapter~\ref{ch:rg_equation}.

%-------------------------------------------------------------------------------
\section{Connection to the Geometric Framework}
\label{sec:qed_geometry}
%-------------------------------------------------------------------------------

We now explicitly connect QED to the geometric framework of Part I.

\subsection{Theory Space}

The coupling space for QED includes $(\alpha, m)$ and gauge-fixing parameters. The essential physics is captured by the flow of $\alpha$:
\begin{equation}
\frac{d\alpha}{ds} = \beta_\alpha(\alpha)
\end{equation}
where $s = \ln(\mu/m)$ is the scale parameter.

\subsection{Fixed Points}

QED has a single perturbative fixed point at $\alpha^* = 0$, the free theory. The stability analysis of Chapter~\ref{ch:fixed_points} reveals the character of perturbations around this fixed point. The coupling $\alpha$ is marginally irrelevant, flowing to $\alpha^* = 0$ in the IR; this explains why electromagnetism appears weakly coupled at low energies. The electron mass $m$ is a relevant perturbation: massive electrons decouple at energies below their mass, leaving only massless photons.

\subsection{Gradient Flow Structure}

In four dimensions, the a-theorem governs RG flows. While the full proof is technical, the decrease of the $a$-anomaly coefficient from UV to IR is consistent with QED flowing toward the free theory.

\subsection{Connections and Scheme Dependence}

The renormalization scheme choice (MS, $\overline{\text{MS}}$, on-shell, etc.) corresponds to a choice of coordinates on theory space. The connection (Chapter~\ref{ch:connections}) ensures that physical observables are scheme-independent.

The Ward identity provides additional structure, constraining the allowed coordinate transformations to preserve gauge invariance.

%-------------------------------------------------------------------------------
\section{Summary}
\label{sec:qed_summary}
%-------------------------------------------------------------------------------

QED demonstrates the five-step RG recipe in its original domain: relativistic quantum field theory. The scale hierarchy (Step 1) extends from the electron mass to the UV cutoff. Coarse-graining (Step 2) integrates out high-momentum virtual particles. Theory space (Step 3) is parametrized by the fine structure constant and electron mass. The beta function (Step 4), equation~\eqref{eq:qed_beta}, describes charge screening by vacuum polarization. Fixed-point analysis (Step 5) reveals that the Gaussian fixed point $\alpha^* = 0$ is an IR attractor.

The chapter has demonstrated every element of the geometric framework in a gauge theory setting. The running of $\alpha$ according to the beta function provides the prototypical example of a running coupling. The Landau pole indicates UV incompleteness, showing that QED cannot be valid at arbitrarily high energies. Anomalous dimensions arise from loop corrections to propagators. Ward identities from gauge invariance constrain the structure of the RG flow, providing an example of how symmetries restrict the geometry of theory space.

The extraordinary precision of QED predictions represents the triumph of the RG approach. The electron anomalous magnetic moment has been computed to five-loop order and agrees with experiment to better than one part in $10^{12}$. This agreement confirms that the abstract geometric framework of Part I, when implemented through systematic perturbative calculations, achieves predictive power unprecedented in the history of physics.

